% !TEX program = xelatex
% !TEX encoding = UTF-8 Unicode

\documentclass[twoside]{article}
\setlength{\oddsidemargin}{0.25 in}
\setlength{\evensidemargin}{-0.25 in}
\setlength{\topmargin}{-0.6 in}
\setlength{\textwidth}{6.5 in}
\setlength{\textheight}{8.5 in}
\setlength{\headsep}{0.75 in}
\setlength{\parindent}{0 in}
\setlength{\parskip}{0.1 in}

%
% ADD PACKAGES here:
%

\usepackage{xeCJK}
\usepackage{amsmath,amsfonts,amsthm,graphicx}
\usepackage{mathtools}
\usepackage{dsfont}
\usepackage{color}   %May be necessary if you want to color links
\usepackage{hyperref}
\usepackage{mleftright}
\hypersetup{
    colorlinks=true, %set true if you want colored links
    linktoc=all,     %set to all if you want both sections and subsections linked
    linkcolor=red,  %choose some color if you want links to stand out
}

%
% The following commands set up the lecnum (lecture number)
% counter and make various numbering schemes work relative
% to the lecture number.
%
\newcounter{lecnum}
\renewcommand{\thepage}{\thelecnum-\arabic{page}}
\renewcommand{\thesection}{\thelecnum.\arabic{section}}
\renewcommand{\theequation}{\thelecnum.\arabic{equation}}
\renewcommand{\thefigure}{\thelecnum.\arabic{figure}}
\renewcommand{\thetable}{\thelecnum.\arabic{table}}

%
% The following macro is used to generate the header.
%
\newcommand{\lecture}[4]{
   \pagestyle{myheadings}
   \thispagestyle{plain}
   \newpage
   \setcounter{lecnum}{#1}
   \setcounter{page}{1}
   \noindent
   \begin{center}
   \framebox{
      \vbox{\vspace{2mm}
    \hbox to 6.28in { {\bf 矩阵论
    \hfill Winter 2017} }
       \vspace{4mm}
       \hbox to 6.28in { {\Large \hfill Lecture #1: #2  \hfill} }
       \vspace{2mm}
       \hbox to 6.28in { {\it Lecturer: #3 \hfill Scribes: #4} }
      \vspace{2mm}}
   }
   \end{center}
   \markboth{Lecture #1: #2}{Lecture #1: #2}
}
%
% Convention for citations is authors' initials followed by the year.
% For example, to cite a paper by Leighton and Maggs you would type
% \cite{LM89}, and to cite a paper by Strassen you would type \cite{S69}.
% (To avoid bibliography problems, for now we redefine the \cite command.)
% Also commands that create a suitable format for the reference list.
\renewcommand{\cite}[1]{[#1]}
\def\beginrefs{\begin{list}%
        {[\arabic{equation}]}{\usecounter{equation}
         \setlength{\leftmargin}{2.0truecm}\setlength{\labelsep}{0.4truecm}%
         \setlength{\labelwidth}{1.6truecm}}}
\def\endrefs{\end{list}}
\def\bibentry#1{\item[\hbox{[#1]}]}

%Use this command for a figure; it puts a figure in wherever you want it.
%usage: \fig{NUMBER}{SPACE-IN-INCHES}{CAPTION}
\newcommand{\fig}[3]{
  \vspace{#2}
  \begin{center}
  Figure \thelecnum.#1:~#3
  \end{center}
}
% Use these for theorems, lemmas, proofs, etc.
\newtheorem{theorem}{定理}[section]
\newtheorem{lemma}{引理}[section]
\newtheorem{proposition}{命题}[section]
\newtheorem{claim}{Claim}[section]
\newtheorem{corollary}{推论}[section]
\newtheorem{definition}{定义}[section]
% \newenvironment{proof}{{\bf Proof:}}{\hfill\rule{2mm}{2mm}}

% **** IF YOU WANT TO DEFINE ADDITIONAL MACROS FOR YOURSELF, PUT THEM HERE:

\newcommand{\norm}[1]{\left\lVert#1\right\rVert}

\begin{document}
%FILL IN THE RIGHT INFO.
%\lecture{**LECTURE-NUMBER**}{**DATE**}{**LECTURER**}{**SCRIBE**}
\lecture{9}{Kronecker积与线性矩阵方程}{Zhitao Liu}{Yusu Pan}
%\footnotetext{These notes are partially based on those of Nigel Mansell.}

% **** YOUR NOTES GO HERE:

% Some general latex examples and examples making use of the
% macros follow.
%**** IN GENERAL, BE BRIEF. LONG SCRIBE NOTES, NO MATTER HOW WELL WRITTEN,
%**** ARE NEVER READ BY ANYBODY.

\section{矩阵的Kronecker积}
\begin{definition}
  (Kronecker积)
  \begin{equation}
    A\otimes B
  \end{equation}
\end{definition}

\begin{itemize}
  \item Kronecker积不满足交换律
\end{itemize}

\begin{theorem}
  (Kronecker积的基本性质)
\end{theorem}

\begin{theorem}
\end{theorem}

\begin{theorem}
  (Kronecker积的特征值, 行列式, 迹)
\end{theorem}

\begin{theorem}
  (Kronecker积的秩)
  \begin{equation}
    rank(A\otimes B) = rank(A) rank(B)
  \end{equation}
\end{theorem}

\paragraph{排列矩阵}
左乘为行交换, 右乘为列交换.
% paragraph 排列矩阵 (end)

\begin{theorem}
  设$A$,$B$分别为$m\times m$和$n\times n$的矩阵, 则存在一个$mn$阶排列矩阵$P$使得
  \begin{equation}
    P^T (A\otimes B) P = B \otimes A
  \end{equation}
\end{theorem}

\paragraph{Kronecker积的幂}
\begin{equation}
  A^{[k]} = A \otimes A \cdots \otimes A
\end{equation}
% paragraph Kronecker积的幂 (end)

\begin{theorem}
  \begin{equation}
    (AB)^{[k]} = A^{[k]} B^{[k]}
  \end{equation}
\end{theorem}
% section 矩阵的Kronecker积 (end)

\section{矩阵的拉直与线性矩阵方程}
\subsection{矩阵的拉直}
\begin{definition}
  (矩阵$A$的列拉直(列展开))
\end{definition}

\paragraph{矩阵$A$的行拉直(行展开)}\label{par:label_name}

% paragraph paragraph name (end)

\begin{theorem}
  \begin{equation}
    vec(ABC) = (C^T \otimes A) vec(B)
  \end{equation}
\end{theorem}

\begin{corollary}
\end{corollary}
% subsection 矩阵的拉直 (end)

\subsection{线性矩阵方程}
\paragraph{线性矩阵方程}\label{par:label_name}
\begin{equation}
  A_1XB_1 + A_2XB_2 + \cdots + A_pXB_p=C
\end{equation}
% paragraph 线性矩阵方程 (end)

\begin{theorem}
  \begin{equation}
    Gx=vec(C)
  \end{equation}
\end{theorem}

\begin{corollary}
  矩阵方程有解的充要条件是, 有唯一解的充要条件是非奇异.
\end{corollary}
% subsection 线性矩阵方程 (end)
% section 矩阵的拉直与线性矩阵方程 (end)

\section{线性方程$AXB=C$与矩阵最佳逼近问题}
\subsection{矩阵方程$AXB=C$}
\begin{theorem}
\end{theorem}
% subsection 矩阵方程$AXB=C$ (end)
\subsection{带约束的矩阵最佳逼近问题}
\paragraph{带约束的矩阵最佳逼近问题}
% paragraph 带约束的矩阵最佳逼近问题 (end)
\begin{theorem}
  设$\mathbf{A}\in \mathbb{C}^{m\times n}$, $\mathbf{B}\in \mathbb{C}^{p\times q}$, $\mathbf{C}\in \mathbb{C}^{m\times q}$, 则$\tilde{X}=(\tilde{x_{ij}} \in \mathbb{C}^{n\times p})$在$S_X$上存在唯一的最佳逼近, 并且
  \begin{equation}
    \hat{X} = A^+CB^+ + \tilde{X} - A^+A\tilde{X}BB^+
  \end{equation}
\end{theorem}
% subsection 带约束的矩阵最佳逼近问题 (end)
% section 线性方程$AXB=C$与矩阵最佳逼近问题 (end)

\section{*}
% section * (end)

\section{*}
% section * (end)

\end{document}
