% !TEX program = xelatex
% !TEX encoding = UTF-8 Unicode

\documentclass[twoside]{article}
\setlength{\oddsidemargin}{0.25 in}
\setlength{\evensidemargin}{-0.25 in}
\setlength{\topmargin}{-0.6 in}
\setlength{\textwidth}{6.5 in}
\setlength{\textheight}{8.5 in}
\setlength{\headsep}{0.75 in}
\setlength{\parindent}{0 in}
\setlength{\parskip}{0.1 in}

%
% ADD PACKAGES here:
%

\usepackage{xeCJK}
\usepackage{amsmath,amsfonts,amsthm,graphicx}
\usepackage{mathtools}
\usepackage{dsfont}
\usepackage{color}   %May be necessary if you want to color links
\usepackage{hyperref}
\usepackage{mleftright}
\hypersetup{
    colorlinks=true, %set true if you want colored links
    linktoc=all,     %set to all if you want both sections and subsections linked
    linkcolor=red,  %choose some color if you want links to stand out
}

%
% The following commands set up the lecnum (lecture number)
% counter and make various numbering schemes work relative
% to the lecture number.
%
\newcounter{lecnum}
\renewcommand{\thepage}{\thelecnum-\arabic{page}}
\renewcommand{\thesection}{\thelecnum.\arabic{section}}
\renewcommand{\theequation}{\thelecnum.\arabic{equation}}
\renewcommand{\thefigure}{\thelecnum.\arabic{figure}}
\renewcommand{\thetable}{\thelecnum.\arabic{table}}

%
% The following macro is used to generate the header.
%
\newcommand{\lecture}[4]{
   \pagestyle{myheadings}
   \thispagestyle{plain}
   \newpage
   \setcounter{lecnum}{#1}
   \setcounter{page}{1}
   \noindent
   \begin{center}
   \framebox{
      \vbox{\vspace{2mm}
    \hbox to 6.28in { {\bf 矩阵论
    \hfill Winter 2017} }
       \vspace{4mm}
       \hbox to 6.28in { {\Large \hfill Lecture #1: #2  \hfill} }
       \vspace{2mm}
       \hbox to 6.28in { {\it Lecturer: #3 \hfill Scribes: #4} }
      \vspace{2mm}}
   }
   \end{center}
   \markboth{Lecture #1: #2}{Lecture #1: #2}
}
%
% Convention for citations is authors' initials followed by the year.
% For example, to cite a paper by Leighton and Maggs you would type
% \cite{LM89}, and to cite a paper by Strassen you would type \cite{S69}.
% (To avoid bibliography problems, for now we redefine the \cite command.)
% Also commands that create a suitable format for the reference list.
\renewcommand{\cite}[1]{[#1]}
\def\beginrefs{\begin{list}%
        {[\arabic{equation}]}{\usecounter{equation}
         \setlength{\leftmargin}{2.0truecm}\setlength{\labelsep}{0.4truecm}%
         \setlength{\labelwidth}{1.6truecm}}}
\def\endrefs{\end{list}}
\def\bibentry#1{\item[\hbox{[#1]}]}

%Use this command for a figure; it puts a figure in wherever you want it.
%usage: \fig{NUMBER}{SPACE-IN-INCHES}{CAPTION}
\newcommand{\fig}[3]{
            \vspace{#2}
            \begin{center}
            Figure \thelecnum.#1:~#3
            \end{center}
    }
% Use these for theorems, lemmas, proofs, etc.
\newtheorem{theorem}{定理}[section]
\newtheorem{lemma}{引理}[section]
\newtheorem{proposition}{命题}[section]
\newtheorem{claim}{Claim}[section]
\newtheorem{corollary}{推论}[section]
\newtheorem{definition}{定义}[section]
% \newenvironment{proof}{{\bf Proof:}}{\hfill\rule{2mm}{2mm}}

% **** IF YOU WANT TO DEFINE ADDITIONAL MACROS FOR YOURSELF, PUT THEM HERE:

\newcommand\E{\mathbb{E}}
\newcommand{\norm}[1]{\left\lVert#1\right\rVert}

\begin{document}
%FILL IN THE RIGHT INFO.
%\lecture{**LECTURE-NUMBER**}{**DATE**}{**LECTURER**}{**SCRIBE**}
\lecture{6}{范数与极限}{Zhitao Liu}{Yusu Pan}
%\footnotetext{These notes are partially based on those of Nigel Mansell.}

% **** YOUR NOTES GO HERE:

% Some general latex examples and examples making use of the
% macros follow.
%**** IN GENERAL, BE BRIEF. LONG SCRIBE NOTES, NO MATTER HOW WELL WRITTEN,
%**** ARE NEVER READ BY ANYBODY.

\section{向量范数}\label{sec:xiang_liang_fan_shu_}
  \begin{definition}
    向量$\alpha$的范数,赋范线性空间
  \end{definition}

  \begin{itemize}
    \item \textbf{向量范数的性质}P5
    \item \textbf{由范数决定的距离}P5
    \item \textbf{1范数,2范数与$\infty$范数}P6
    \item \textbf{$p$范数}P7
  \end{itemize}

  \begin{lemma}
    如果实数$p>1,q>1$且$\frac{1}{p}+\frac{1}{q}=1$,则对任意非负实数$a,b$有$ab\le \frac{a^p}{p}+\frac{b^q}{q}$
  \end{lemma}

  \begin{theorem}
    (Holder不等式)
    P9
  \end{theorem}

  \begin{theorem}
    (Minkowski不等式)
    P9
  \end{theorem}

  \begin{theorem}
    ($p$范数趋向于$\infty$范数)
    对任意向量$x=(x_1, \cdots, x_n)^T \in \mathds{C}^n, 1\le p < \infty , \norm{x}_p=\biggl(\sum\biggr)$
    P10
  \end{theorem}

  \begin{theorem}
    (由已知范数构造新的向量范数)
    设$\norm{\cdot}_{\beta}$是$\mathds{C}^m$上的向量范数,$\mathbf{A}\in \mathds{C}^{m\times n}$且$rank(\mathbf{A})=n$,则由
    \begin{equation*}
      \norm{x}_\alpha = \norm{\mathbf{A}x}_\beta , x\in \mathds{C}^n
    \end{equation*}
    所定义的$\norm{\cdot}_{\alpha}$是$\mathds{C}^n$上的向量范数.
  \end{theorem}

  \begin{theorem}
    设$\mathbf{V}$是数域$\mathds{P}$上的$n$维线性空间,
    P11
  \end{theorem}

  \begin{definition}
    (范数的等价性) P13
  \end{definition}

  \begin{theorem}
    (向量范数的连续性)
    设$\norm{\cdot}$是数域$\mathds{P}$上的$n$维线性空间$\mathbf{V}$上的任一向量范数,$\epsilon_1,\cdots \epsilon_n$为i$\mathbf{V}$的一组基,$\mathbf{V}$中的任一向量$\alpha$可以唯一地表示为$\alpha=\sum^n_{i=1} x_i \epsilon_i, x=(x_1,\cdots,x_n)^T \in \mathds{P}^n$,则$\norm{\alpha}$是$x_1,\cdots x_n$的连续函数.
  \end{theorem}

  \begin{theorem}
    有限维线性空间$\mathbf{V}$上的任意两个向量范数都是等价的.
  \end{theorem}

  \begin{definition}
    (向量序列的收敛性) 收敛,极限,发散 P16
  \end{definition}

  向量收敛性的性质 (P17)
  \begin{itemize}
    \item 1
    \item 2
  \end{itemize}

  \begin{theorem}
    (向量收敛的充要条件) P18
  \end{theorem}
% section 向量范数 (end)

\section{矩阵范数}\label{sec:ju_zhen_fan_shu_}
  \subsection{矩阵范数的定义与等价性定理}\label{sub:ju_zhen_fan_shu_de_ding_yi_yu_deng_jie_xing_ding_li_}
    \begin{definition}
      (矩阵范数的定义) P21
      \begin{itemize}
        \item 非负性
        \item 齐次性
        \item 三角不等式
      \end{itemize}
    \end{definition}

    \begin{theorem}
      (矩阵范数等价性定理) P23
    \end{theorem}
  % subsection 矩阵范数的定义与等价性定理 (end)

  \subsection{相容矩阵范数}\label{sub:xiang_rong_ju_zhen_fan_shu_}
    相容性条件 P24

    \begin{equation*}
      \norm{\mathbf{AB}} \le \norm{\mathbf{A}} \norm{\mathbf{B}}
    \end{equation*}
    \begin{itemize}
      \item Fribenius范数$\norm{\cdot}_F$具备相容性条件
      \item 矩阵范数$\norm{\cdot}_1'$具备相容性条件
      \item 并非所有矩阵范数都具备相容性条件,如矩阵范数$\norm{\cdot}_\infty'$
    \end{itemize}

    \begin{definition}
      (矩阵范数与向量范数的相容性定义) P26
    \end{definition}

    \begin{theorem}
      设$\norm{\cdot}$是$\mathds{C}^{n\times n}$上的相容矩阵范数,则在$\mathds{C}^{n}$上存在与$\norm{\cdot}$相容的向量范数.
    \end{theorem}

    \begin{theorem}
      P28
    \end{theorem}
    \begin{itemize}
      \item 可用于谱半径的估计:$\rho (\mathbf{A}) = \max _i |\lambda_i| \le \norm{\mathbf{A}}$
    \end{itemize}
  % subsection 相容矩阵范数 (end)

  \subsection{算子范数}\label{sub:suan_zi_fan_shu_}
    \begin{lemma}
      (有界闭集) P29
    \end{lemma}

    \begin{lemma}
      P29
    \end{lemma}

    \begin{theorem}
      (由向量范数导出矩阵范数的定义) 设$\norm{\cdot}_\mu$和$\norm{\cdot}_\nu$分别是$\mathds{C}^m$和$\mathds{C}^n$上的两个向量范数,对$\mathbf{A}\in \mathds{C}^{m\times n}$,令

      \begin{equation*}
        \norm{\mathbf{A}}_{\mu, \nu} = \max_{\norm{x}_\nu =1} \norm{\mathbf{A}x}_\mu
      \end{equation*}

      则$\norm{\cdot}_{\mu, \nu}$是$\mathds{C}^{m\times n}$上的矩阵范数,并且$\norm{\cdot}_{\mu}$,$\norm{\cdot}_{\nu}$和$\norm{\cdot}_{\mu, \nu}$相容.
    \end{theorem}

    \begin{definition}
      (算子范数) P32
    \end{definition}

    \begin{theorem}
      (算子范数之间的相容性) P33
    \end{theorem}
  % subsection 算子范数 (end)

  \subsection{常见矩阵级数}\label{sub:chang_jian_ju_zhen_ji_shu_}
    \subsubsection{p算子范数$\norm{A}_p$}\label{ssub:psuan_zi_fan_shu_}
      \begin{theorem}
        (列和范数,谱范数,行和范数) P36
      \end{theorem}
    % subsubsection p算子范数 (end)

    \subsubsection{Frobenius范数$\norm{A}_F$}\label{ssub:frobeniusfan_shu_}
      P38
      \begin{theorem}
        P38
      \end{theorem}
    % subsubsection Frobenius范数 (end)

    \subsubsection{谱范数$\norm{A}_2$}\label{ssub:pu_fan_shu_}
      \begin{theorem}
        P39
      \end{theorem}
    % subsubsection 谱范数 (end)
  % subsection 常见矩阵级数 (end)
% section 矩阵范数 (end)

\section{矩阵序列与矩阵级数}\label{sec:ju_zhen_xu_lie_yu_ju_zhen_ji_shu_}
  \subsection{矩阵序列}\label{sub:ju_zhen_xu_lie_}
    \begin{definition}
      P41
    \end{definition}

    \begin{theorem}
      P42
    \end{theorem}

    矩阵序列的极限运算的性质 P43

    \begin{theorem}
      P44
    \end{theorem}

    \begin{theorem}
      P46
    \end{theorem}

    \begin{theorem}
      P47
    \end{theorem}
  % subsection 矩阵序列 (end)

  \subsection{矩阵级数}\label{sub:ju_zhen_ji_shu_}
    \begin{definition}
      (矩阵级数的定义)
      P48
    \end{definition}

    \textbf{矩阵级数的性质} P49

    \begin{definition}
      (矩阵级数绝对收敛的定义) P50
    \end{definition}

    \begin{theorem}
      (矩阵级数绝对收敛的充要条件) P50
    \end{theorem}

    \begin{definition}
      (矩阵幂级数的定义) P51
    \end{definition}

    \begin{theorem}
      (矩阵幂级数绝对收敛的充分条件) P51
    \end{theorem}

    \begin{corollary}
      P52
    \end{corollary}

    \begin{theorem}
      P52
    \end{theorem}

    \begin{corollary}
      (幂级数收敛与矩阵幂级数收敛)
      P53
    \end{corollary}

    \begin{theorem}
      P53
    \end{theorem}

    \begin{theorem}
      (矩阵幂级数收敛的充要条件)
      P54

      可相比于数项级数

      \begin{equation*}
          \sum_{k=0}^\infty L^k = \frac{1}{1-L}, 0<L<1
      \end{equation*}

    \end{theorem}

  % subsection 矩阵级数 (end)
% section 矩阵序列与矩阵级数 (end)

\section{矩阵扰动分析}\label{sec:ju_zhen_rao_dong_fen_xi_}
  (不在考试范围之内)
  \subsection{矩阵逆的扰动分析}
    \begin{theorem}
      (矩阵逆的扰动上界)
      P59
    \end{theorem}

    \begin{theorem}
      P60
    \end{theorem}

    \begin{lemma}
      (条件数的定义)
      P61
    \end{lemma}
  % subsection 矩阵逆的扰动分析 (end)


  \subsection{线性方程组解的扰动分析}
    \begin{theorem}
      (线性方程组解的扰动上界)
      P63
    \end{theorem}
  % subsection 线性方程组解的扰动分析 (end)
% section* 矩阵扰动分析 (end)


\section*{作业}\label{sec:zuo_ye_}
  P198 Q10
% section* 作业 (end)

\end{document}




