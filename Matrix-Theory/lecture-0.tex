% !TEX program = xelatex
% !TEX encoding = UTF-8 Unicode

\documentclass[twoside]{article}
\setlength{\oddsidemargin}{0.25 in}
\setlength{\evensidemargin}{-0.25 in}
\setlength{\topmargin}{-0.6 in}
\setlength{\textwidth}{6.5 in}
\setlength{\textheight}{8.5 in}
\setlength{\headsep}{0.75 in}
\setlength{\parindent}{0 in}
\setlength{\parskip}{0.1 in}

%
% ADD PACKAGES here:
%

\usepackage{xeCJK}
\usepackage{amsmath,amsfonts,amsthm,graphicx}
\usepackage{mathtools}
\usepackage{dsfont}
\usepackage{color}   %May be necessary if you want to color links
\usepackage{hyperref}
\usepackage{mleftright}
\hypersetup{
    colorlinks=true, %set true if you want colored links
    linktoc=all,     %set to all if you want both sections and subsections linked
    linkcolor=red,  %choose some color if you want links to stand out
}

%
% The following commands set up the lecnum (lecture number)
% counter and make various numbering schemes work relative
% to the lecture number.
%
\newcounter{lecnum}
\renewcommand{\thepage}{\thelecnum-\arabic{page}}
\renewcommand{\thesection}{\thelecnum.\arabic{section}}
\renewcommand{\theequation}{\thelecnum.\arabic{equation}}
\renewcommand{\thefigure}{\thelecnum.\arabic{figure}}
\renewcommand{\thetable}{\thelecnum.\arabic{table}}

%
% The following macro is used to generate the header.
%
\newcommand{\lecture}[4]{
   \pagestyle{myheadings}
   \thispagestyle{plain}
   \newpage
   \setcounter{lecnum}{#1}
   \setcounter{page}{1}
   \noindent
   \begin{center}
   \framebox{
      \vbox{\vspace{2mm}
    \hbox to 6.28in { {\bf 矩阵论
    \hfill Winter 2017} }
       \vspace{4mm}
       \hbox to 6.28in { {\Large \hfill Lecture #1: #2  \hfill} }
       \vspace{2mm}
       \hbox to 6.28in { \textit{Lecturer: #3 \hfill Scribes: #4} }
      \vspace{2mm}}
   }
   \end{center}
   \markboth{Lecture #1: #2}{Lecture #1: #2}
}
% Use these for theorems, lemmas, proofs, etc.
\newtheorem{theorem}{定理}[section]
\newtheorem{lemma}{引理}[section]
\newtheorem{proposition}{命题}[section]
\newtheorem{claim}{Claim}[section]
\newtheorem{corollary}{推论}[section]
\newtheorem{definition}{定义}[section]
% \newenvironment{proof}{{\bf Proof:}}{\hfill\rule{2mm}{2mm}}

% **** IF YOU WANT TO DEFINE ADDITIONAL MACROS FOR YOURSELF, PUT THEM HERE:

\newcommand{\norm}[1]{\left\lVert#1\right\rVert}

\begin{document}
%FILL IN THE RIGHT INFO.
%\lecture{**LECTURE-NUMBER**}{**DATE**}{**LECTURER**}{**SCRIBE**}
\lecture{0}{预备知识: 群, 环, 域及线性空间}{Jun Wu}{Yusu Pan}
%\footnotetext{These notes are partially based on those of Nigel Mansell.}

% **** YOUR NOTES GO HERE:

% Some general latex examples and examples making use of the
% macros follow.
%**** IN GENERAL, BE BRIEF. LONG SCRIBE NOTES, NO MATTER HOW WELL WRITTEN,
%**** ARE NEVER READ BY ANYBODY.

\begin{definition}
  (二元运算/代数运算)
\end{definition}

\begin{definition}
  (半群) 满足结合律
  \begin{equation}
    a\circ (b\circ c) = (a\circ b) \circ c
  \end{equation}
\end{definition}

\begin{definition}
  (单位元)
  \begin{equation}
    a\circ e = e \circ a = a
  \end{equation}
\end{definition}

\paragraph{可逆}
\begin{equation}
  \exists a'\in S, a'\circ a = a\circ a=e
\end{equation}
% paragraph 可逆 (end)

\begin{definition}
  (群)
  \begin{itemize}
    \item 半群 (结合律)
    \item 单位元
    \item 可逆
  \end{itemize}
\end{definition}

\begin{definition}
  (交换半群)
  \begin{equation}
    a\circ b = b\circ a
  \end{equation}
\end{definition}

\begin{definition}
  (交换群/Abel群)
\end{definition}
\begin{itemize}
  \item 加法, 交换群又称加法群$(S, +)$
  \item 零元, 非零元
  \item 乘法: 半群的运算
\end{itemize}

\begin{definition}
  (环)
  \begin{itemize}
    \item $(S, +)$是交换群
    \item $(S, \cdot)$是半群
    \item 对$+$分配律成立
      \begin{equation}
        a(b+c)=ab+ac, (b+c)a=ba+ca
      \end{equation}
  \end{itemize}
\end{definition}
\begin{itemize}
  \item 有单位元的环
  \item 交换环
\end{itemize}

\begin{definition}
  (域)
  \begin{itemize}
    \item $(S,+,\cdot)$是有单位元的交换环
    \item $\forall a\ne 0, \exists a'\in S, a'a=1, aa'=1$
  \end{itemize}
\end{definition}

\begin{definition}
  (域$\mathbb{P}$上的线性空间/向量空间$S$)
\end{definition}
\begin{itemize}
  \item 零向量
\end{itemize}

\begin{definition}
  (线性相关与线性无关)
\end{definition}

\begin{definition}
  (基)
\end{definition}

\begin{definition}
  (维数)
\end{definition}
\end{document}

