% !TEX program = xelatex
% !TEX encoding = UTF-8 Unicode

\documentclass[twoside]{article}
\setlength{\oddsidemargin}{0.25 in}
\setlength{\evensidemargin}{-0.25 in}
\setlength{\topmargin}{-0.6 in}
\setlength{\textwidth}{6.5 in}
\setlength{\textheight}{8.5 in}
\setlength{\headsep}{0.75 in}
\setlength{\parindent}{0 in}
\setlength{\parskip}{0.1 in}

%
% ADD PACKAGES here:
%

\usepackage{xeCJK}
\usepackage{amsmath,amsfonts,amsthm,graphicx}
\usepackage{mathtools}
\usepackage{dsfont}
\usepackage{color}   %May be necessary if you want to color links
\usepackage{hyperref}
\usepackage{mleftright}
\hypersetup{
    colorlinks=true, %set true if you want colored links
    linktoc=all,     %set to all if you want both sections and subsections linked
    linkcolor=red,  %choose some color if you want links to stand out
}

%
% The following commands set up the lecnum (lecture number)
% counter and make various numbering schemes work relative
% to the lecture number.
%
\newcounter{lecnum}
\renewcommand{\thepage}{\thelecnum-\arabic{page}}
\renewcommand{\thesection}{\thelecnum.\arabic{section}}
\renewcommand{\theequation}{\thelecnum.\arabic{equation}}
\renewcommand{\thefigure}{\thelecnum.\arabic{figure}}
\renewcommand{\thetable}{\thelecnum.\arabic{table}}

%
% The following macro is used to generate the header.
%
\newcommand{\lecture}[4]{
   \pagestyle{myheadings}
   \thispagestyle{plain}
   \newpage
   \setcounter{lecnum}{#1}
   \setcounter{page}{1}
   \noindent
   \begin{center}
   \framebox{
      \vbox{\vspace{2mm}
    \hbox to 6.28in { {\bf 矩阵论
    \hfill Winter 2017} }
       \vspace{4mm}
       \hbox to 6.28in { {\Large \hfill Lecture #1: #2  \hfill} }
       \vspace{2mm}
       \hbox to 6.28in { {\it Lecturer: #3 \hfill Scribes: #4} }
      \vspace{2mm}}
   }
   \end{center}
   \markboth{Lecture #1: #2}{Lecture #1: #2}
}
%
% Convention for citations is authors' initials followed by the year.
% For example, to cite a paper by Leighton and Maggs you would type
% \cite{LM89}, and to cite a paper by Strassen you would type \cite{S69}.
% (To avoid bibliography problems, for now we redefine the \cite command.)
% Also commands that create a suitable format for the reference list.
\renewcommand{\cite}[1]{[#1]}
\def\beginrefs{\begin{list}%
        {[\arabic{equation}]}{\usecounter{equation}
         \setlength{\leftmargin}{2.0truecm}\setlength{\labelsep}{0.4truecm}%
         \setlength{\labelwidth}{1.6truecm}}}
\def\endrefs{\end{list}}
\def\bibentry#1{\item[\hbox{[#1]}]}

%Use this command for a figure; it puts a figure in wherever you want it.
%usage: \fig{NUMBER}{SPACE-IN-INCHES}{CAPTION}
\newcommand{\fig}[3]{
  \vspace{#2}
  \begin{center}
  Figure \thelecnum.#1:~#3
  \end{center}
}
% Use these for theorems, lemmas, proofs, etc.
\newtheorem{theorem}{定理}[section]
\newtheorem{lemma}{引理}[section]
\newtheorem{proposition}{命题}[section]
\newtheorem{claim}{Claim}[section]
\newtheorem{corollary}{推论}[section]
\newtheorem{definition}{定义}[section]
% \newenvironment{proof}{{\bf Proof:}}{\hfill\rule{2mm}{2mm}}

% **** IF YOU WANT TO DEFINE ADDITIONAL MACROS FOR YOURSELF, PUT THEM HERE:

\newcommand{\norm}[1]{\left\lVert#1\right\rVert}

\begin{document}
%FILL IN THE RIGHT INFO.
%\lecture{**LECTURE-NUMBER**}{**DATE**}{**LECTURER**}{**SCRIBE**}
\lecture{8}{广义逆矩阵}{Zhitao Liu}{Yusu Pan}
%\footnotetext{These notes are partially based on those of Nigel Mansell.}

% **** YOUR NOTES GO HERE:

% Some general latex examples and examples making use of the
% macros follow.
%**** IN GENERAL, BE BRIEF. LONG SCRIBE NOTES, NO MATTER HOW WELL WRITTEN,
%**** ARE NEVER READ BY ANYBODY.

\section*{考试与作业}
\paragraph{考试}
\begin{itemize}
    \item 题型:选择, 填空, 大题
    \item 共9题
    \item 大题有证明题,主要是第六章(注意)与第八章
    \item 群环域等预备知识均在期末考试范围内
    \item 注意掌握课后习题的类型, 题型是一致的(例如$e^{At}$以及$A^+$的计算), 数字会变化 
    \item 可以带计算器
\end{itemize}

\paragraph{作业}
\begin{itemize}
    \item P247Q9(1)(3)
    \item P247Q15(1)
\end{itemize}

\section{广义逆矩阵的概念}
\paragraph{Penrose方程}
\begin{equation*}
    \begin{aligned}
        & AGA = A \\
        & GAG = G \\
        & (AG)^T = AG \
        & (GA)^T = GA
    \end{aligned}
\end{equation*}

\begin{definition}
(广义逆矩阵的定义)
\end{definition}

广义逆的种类

\section{广义逆矩阵$A^-$与线性方程组的解}
\begin{equation*}
    \begin{aligned}
        & AGA = A \\
    \end{aligned}
\end{equation*}

\begin{theorem}
($A^-$的计算)
\end{theorem}
\begin{proof}
\begin{equation}
    \begin{aligned}
        AGA &= AQ[\cdots]PA \\
        & = P{^-1} [\cdots][\cdots][\cdots]Q^{-1} \\
        & = P^{-1} [\cdots] Q^{-1}\\
        & =A
    \end{aligned}
\end{equation}
\end{proof}
注意$A^-$并不唯一

\begin{theorem}
($A^-$的性质)
\end{theorem}
\begin{proof}
\begin{equation}
    \begin{aligned}
        & BGB = B \\
        & G =Q^{-1} A^- P^{-1} \\
        & BQ^{-1}AP^{-1}B = PAQQ^{-1}AP^{-1}PAQ=B
    \end{aligned}
\end{equation}
\end{proof}
注意$A^-$与$P^{-1}$在写法上的区别

\begin{theorem}
(相容方程组的解)
\end{theorem}
相容方程组$Ax=b$即为有解的方程组, $rank([A|b]) = rank(A)$, 解不一定唯一.

\begin{theorem}
(Penrose定理/矩阵方程$AXB=C$的通解)

设$A$,$B$,$C$分别为$m\times n$, $p\times q$, $m\times q$矩阵, 则矩阵方程
\begin{equation*}
    AXB=C
\end{equation*}
有解的充分必要条件是
\begin{equation*}
    AA^-CB^-B=C
\end{equation*}
并且在有解的情况下, 其通解为
\begin{equation*}
    X=A^-CB^-+Y-A^-AYBB^-
\end{equation*}
其中$Y\in \mathds{R}^{n\times p}$是任意的矩阵
\end{theorem}

\begin{theorem}
(线性方程组$Ax=b$的通解)
\end{theorem}

\begin{theorem}
($A\{1\}$的通式)
\end{theorem}

\section{极小范数广义逆$A_m^-$与线性方程组的极小范数解}
\begin{equation*}
    \begin{aligned}
        & AGA = A \\
        & (GA)^T = GA
    \end{aligned}
\end{equation*}

\begin{theorem}
($G\in A\{1,4\}$的充要条件1)
\end{theorem}

\begin{theorem}
($G\in A\{1,4\}$的充要条件2)
\end{theorem}

\begin{theorem}
($A\{1,4\}$的通式1)
\end{theorem}

\begin{theorem}
($A\{1,4\}$的通式2)
\end{theorem}

\begin{theorem}
(相容方程组$Ax=b$的极小范数解)
\end{theorem}
\begin{proof}
\begin{equation*}
    \begin{aligned}
        & Ax=b, & A\in\mathds{R}^{m\times n} \\
        & x = Gb + (I-GA)y, & AGA=A \\
        & x^* = G^*b + (I-G^*A)y, & AG^*A=A, (G^*A)^T=G^*A
    \end{aligned}
\end{equation*}
\end{proof}

\begin{theorem}
(相容方程组$Ax=b$的极小范数解的唯一性)
\end{theorem}

\section{最小二乘广义逆$A^-_l$与矛盾方程组的最小二乘解}
\begin{equation*}
    \begin{aligned}
        & AGA = A \\
        & (AG)^T = AG
    \end{aligned}
\end{equation*}

\begin{theorem}
($G\in A\{1,3\}$的充要条件1)
\end{theorem}

\begin{theorem}
($G\in A\{1,3\}$的充要条件2)
\end{theorem}

\begin{theorem}
(最小二乘逆1)
\end{theorem}

\begin{theorem}
(最小二乘逆2)
\end{theorem}

\paragraph{线性最小二乘问题}
如果线性方程组$Ax=b$不相容, 则它没有通常意义下的解,残量不等于零. 求这样的解, 使它的残量范数最小.

即求在上的最佳逼近
满足上式的...

\begin{theorem}
(线性方程组$Ax=b$的最小二乘解)
\end{theorem}

\begin{theorem}
不相容线性方程组$Ax=b$的最小二乘解必为相容线性方程组$A^T Ax=A^T b$的解, 反之亦然.
\end{theorem}

\begin{corollary}
(线性方程组$Ax=b$的最小二乘解的唯一性)
当$A$为列满秩时, 不相容方程组$Ax=b$的最小二乘解是唯一的.
\end{corollary}

\begin{theorem}
(线性方程组$Ax=b$的最小二乘解的通式)
\end{theorem}

\section{广义逆矩阵$A^+$与线性方程组的极小最小二乘解}
\begin{equation*}
    \begin{aligned}
        & AGA = A \\
        & GAG = G \\
        & (AG)^T = AG \\
        & (GA)^T = GA
    \end{aligned}
\end{equation*}

\begin{theorem}
($A^+$的唯一性)
\end{theorem}

\begin{theorem}
(通过满秩分解求$A^+$)
设$A$是$m\times n$, 其满秩分解为
\begin{equation*}
    A=BC
\end{equation*}
其中, 则
\begin{equation*}
    A^+ = C^T (CC^T)^{-1} (B^TB)^{-1} B^T
\end{equation*}
\end{theorem}

\begin{theorem}
($A^+$的性质)
\end{theorem}

\begin{theorem}
(线性方程组$Ax=b$的通解)
\end{theorem}

\begin{theorem}
(不相容线性方程组$Ax=b$的最小二乘通解)
\end{theorem}

\begin{theorem}
(不相容线性方程组$Ax=b$的极小最小二乘解)
\end{theorem}

\end{document}

