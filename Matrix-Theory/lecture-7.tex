% !TEX program = xelatex
% !TEX encoding = UTF-8 Unicode

\documentclass[twoside]{article}
\setlength{\oddsidemargin}{0.25 in}
\setlength{\evensidemargin}{-0.25 in}
\setlength{\topmargin}{-0.6 in}
\setlength{\textwidth}{6.5 in}
\setlength{\textheight}{8.5 in}
\setlength{\headsep}{0.75 in}
\setlength{\parindent}{0 in}
\setlength{\parskip}{0.1 in}

%
% ADD PACKAGES here:
%

\usepackage{xeCJK}
\usepackage{amsmath,amsfonts,amsthm,graphicx}
\usepackage{mathtools}
\usepackage{dsfont}
\usepackage{color}   %May be necessary if you want to color links
\usepackage{hyperref}
\usepackage{mleftright}
\hypersetup{
    colorlinks=true, %set true if you want colored links
    linktoc=all,     %set to all if you want both sections and subsections linked
    linkcolor=red,  %choose some color if you want links to stand out
}

%
% The following commands set up the lecnum (lecture number)
% counter and make various numbering schemes work relative
% to the lecture number.
%
\newcounter{lecnum}
\renewcommand{\thepage}{\thelecnum-\arabic{page}}
\renewcommand{\thesection}{\thelecnum.\arabic{section}}
\renewcommand{\theequation}{\thelecnum.\arabic{equation}}
\renewcommand{\thefigure}{\thelecnum.\arabic{figure}}
\renewcommand{\thetable}{\thelecnum.\arabic{table}}

%
% The following macro is used to generate the header.
%
\newcommand{\lecture}[4]{
   \pagestyle{myheadings}
   \thispagestyle{plain}
   \newpage
   \setcounter{lecnum}{#1}
   \setcounter{page}{1}
   \noindent
   \begin{center}
   \framebox{
      \vbox{\vspace{2mm}
    \hbox to 6.28in { {\bf 矩阵论
    \hfill Winter 2017} }
       \vspace{4mm}
       \hbox to 6.28in { {\Large \hfill Lecture #1: #2  \hfill} }
       \vspace{2mm}
       \hbox to 6.28in { {\it Lecturer: #3 \hfill Scribes: #4} }
      \vspace{2mm}}
   }
   \end{center}
   \markboth{Lecture #1: #2}{Lecture #1: #2}
}
%
% Convention for citations is authors' initials followed by the year.
% For example, to cite a paper by Leighton and Maggs you would type
% \cite{LM89}, and to cite a paper by Strassen you would type \cite{S69}.
% (To avoid bibliography problems, for now we redefine the \cite command.)
% Also commands that create a suitable format for the reference list.
\renewcommand{\cite}[1]{[#1]}
\def\beginrefs{\begin{list}%
        {[\arabic{equation}]}{\usecounter{equation}
         \setlength{\leftmargin}{2.0truecm}\setlength{\labelsep}{0.4truecm}%
         \setlength{\labelwidth}{1.6truecm}}}
\def\endrefs{\end{list}}
\def\bibentry#1{\item[\hbox{[#1]}]}

%Use this command for a figure; it puts a figure in wherever you want it.
%usage: \fig{NUMBER}{SPACE-IN-INCHES}{CAPTION}
\newcommand{\fig}[3]{
  \vspace{#2}
  \begin{center}
  Figure \thelecnum.#1:~#3
  \end{center}
}
% Use these for theorems, lemmas, proofs, etc.
\newtheorem{theorem}{定理}[section]
\newtheorem{lemma}{引理}[section]
\newtheorem{proposition}{命题}[section]
\newtheorem{claim}{Claim}[section]
\newtheorem{corollary}{推论}[section]
\newtheorem{definition}{定义}[section]
% \newenvironment{proof}{{\bf Proof:}}{\hfill\rule{2mm}{2mm}}

% **** IF YOU WANT TO DEFINE ADDITIONAL MACROS FOR YOURSELF, PUT THEM HERE:

\newcommand\A{\mathbf{A}}
\newcommand\C{\mathds{C}}
\newcommand\R{\mathds{R}}
\newcommand{\norm}[1]{\left\lVert#1\right\rVert}

\begin{document}
%FILL IN THE RIGHT INFO.
%\lecture{**LECTURE-NUMBER**}{**DATE**}{**LECTURER**}{**SCRIBE**}
\lecture{7}{矩阵函数与矩阵值函数}{Zhitao Liu}{Yusu Pan}
%\footnotetext{These notes are partially based on those of Nigel Mansell.}

% **** YOUR NOTES GO HERE:

% Some general latex examples and examples making use of the
% macros follow.
%**** IN GENERAL, BE BRIEF. LONG SCRIBE NOTES, NO MATTER HOW WELL WRITTEN,
%**** ARE NEVER READ BY ANYBODY.

\section*{作业}\label{sec:homework}
(Page 230) Chapter 7 Probelm 1\&9

考试形式: 半开卷
% section* 作业 (end)

\section{矩阵函数}
\subsection{矩阵函数的幂级数表示}
\begin{definition}
  (矩阵函数的定义)
\end{definition}

\begin{itemize}
  \item 矩阵指数函数
  \item 矩阵三角函数及其一般性质
\end{itemize}

\begin{theorem}
  ($AB=BA$时矩阵指数函数的性质)
\end{theorem}

计算$f(\mathbf{A})$
\begin{itemize}
  \item 利用相似对角化
  \item 利用Jordan型
  \item 数项级数求和法: 将矩阵幂级数的求和问题转化为$m$个数项级数的求和问题.
\end{itemize}
% subsection 矩阵函数的幂级数表示 (end)
\subsection{矩阵函数的另一种定义}
\begin{definition}
\end{definition}
% subsection 矩阵函数的另一种定义 (end)
% section 矩阵函数 (end)

\section{矩阵值函数}
\subsection{矩阵值函数的定义}
\begin{definition}
  (矩阵值函数的定义)
\end{definition}
% subsection 矩阵值函数的定义 (end)

\subsection{矩阵值函数的分析运算}
\begin{definition}
  (矩阵值函数的导数)
\end{definition}

\begin{itemize}
  \item 矩阵值函数的导数运算的性质
\end{itemize}

\begin{theorem}
  (矩阵值函数的逆的导数)
\end{theorem}

\begin{itemize}
  \item 矩阵值函数的积分运算的性质
\end{itemize}

\begin{definition}
  (数量函数对矩阵变量的导数)
\end{definition}

\begin{definition}
  (矩阵值函数对矩阵变量的导数)
\end{definition}
% subsection 矩阵值函数的分析运算 (end)
% section 矩阵值函数 (end)

\section{矩阵值函数在微分方程组中的应用}
微分方程组的初值问题
\begin{equation}
  \left\{
    \begin{aligned}
      \frac{dx(t)}{dt} &= \mathbf{A}(t)x(t) + f(t) \\
      x(t_o) &= x_0
    \end{aligned}
  \right.
\end{equation}

通解为
\begin{equation}
  x(t) = e^{\mathbf{A}(t-t_o)} x_0 + e^{\mathbf{A}t}\int^t_{t_0} 
\end{equation}

\begin{definition}
  (渐进稳定)
\end{definition}

\begin{definition}
  (稳定矩阵)
\end{definition}

\begin{theorem}
  (渐进稳定的充要条件)
\end{theorem}
% section 矩阵值函数在微分方程组中的应用 (end)

\end{document}
