% !TEX program = xelatex
% !TEX encoding = UTF-8 Unicode

\documentclass[twoside]{article}
\setlength{\oddsidemargin}{0.25 in}
\setlength{\evensidemargin}{-0.25 in}
\setlength{\topmargin}{-0.6 in}
\setlength{\textwidth}{6.5 in}
\setlength{\textheight}{8.5 in}
\setlength{\headsep}{0.75 in}
\setlength{\parindent}{0 in}
\setlength{\parskip}{0.1 in}

%
% ADD PACKAGES here:
%

\usepackage{xeCJK}
\usepackage{amsmath,amsfonts,amsthm,graphicx}
\usepackage{mathtools}
\usepackage{color}   %May be necessary if you want to color links
\usepackage{hyperref}
\usepackage{mleftright}
\hypersetup{
    colorlinks=true, %set true if you want colored links
    linktoc=all,     %set to all if you want both sections and subsections linked
    linkcolor=red,  %choose some color if you want links to stand out
}

%
% The following commands set up the lecnum (lecture number)
% counter and make various numbering schemes work relative
% to the lecture number.
%
\newcounter{lecnum}
\renewcommand{\thepage}{\thelecnum-\arabic{page}}
\renewcommand{\thesection}{\thelecnum.\arabic{section}}
\renewcommand{\theequation}{\thelecnum.\arabic{equation}}
\renewcommand{\thefigure}{\thelecnum.\arabic{figure}}
\renewcommand{\thetable}{\thelecnum.\arabic{table}}

%
% The following macro is used to generate the header.
%
\newcommand{\lecture}[4]{
   \pagestyle{myheadings}
   \thispagestyle{plain}
   \newpage
   \setcounter{lecnum}{#1}
   \setcounter{page}{1}
   \noindent
   \begin{center}
   \framebox{
      \vbox{\vspace{2mm}
    \hbox to 6.28in { {\bf 科学与工程计算方法
        \hfill Winter 2017} }
       \vspace{4mm}
       \hbox to 6.28in { {\Large \hfill Lecture #1: #2  \hfill} }
       \vspace{2mm}
       \hbox to 6.28in { {\it Lecturer: #3 \hfill Scribes: #4} }
      \vspace{2mm}}
   }
   \end{center}
   \markboth{Lecture #1: #2}{Lecture #1: #2}
}
%
% Convention for citations is authors' initials followed by the year.
% For example, to cite a paper by Leighton and Maggs you would type
% \cite{LM89}, and to cite a paper by Strassen you would type \cite{S69}.
% (To avoid bibliography problems, for now we redefine the \cite command.)
% Also commands that create a suitable format for the reference list.
\renewcommand{\cite}[1]{[#1]}
\def\beginrefs{\begin{list}%
        {[\arabic{equation}]}{\usecounter{equation}
         \setlength{\leftmargin}{2.0truecm}\setlength{\labelsep}{0.4truecm}%
         \setlength{\labelwidth}{1.6truecm}}}
\def\endrefs{\end{list}}
\def\bibentry#1{\item[\hbox{[#1]}]}

%Use this command for a figure; it puts a figure in wherever you want it.
%usage: \fig{NUMBER}{SPACE-IN-INCHES}{CAPTION}
\newcommand{\fig}[3]{
    \vspace{#2}
    \begin{center}
        Figure \thelecnum.#1:~#3
    \end{center}
}
% Use these for theorems, lemmas, proofs, etc.
\newtheorem{theorem}{定理}[section]
\newtheorem{algo}{算法}[section]
\newtheorem{lemma}{引理}[section]
\newtheorem{proposition}{命题}[section]
\newtheorem{claim}{Claim}[section]
\newtheorem{corollary}{推论}[section]
\newtheorem{definition}{定义}[section]
% \newenvironment{proof}{{\bf Proof:}}{\hfill\rule{2mm}{2mm}}

% **** IF YOU WANT TO DEFINE ADDITIONAL MACROS FOR YOURSELF, PUT THEM HERE:

\begin{document}
%FILL IN THE RIGHT INFO.
%\lecture{**LECTURE-NUMBER**}{**DATE**}{**LECTURER**}{**SCRIBE**}
\lecture{1}{绪论}{Zhitao Liu}{Yusu Pan}
%\footnotetext{These notes are partially based on those of Nigel Mansell.}

% **** YOUR NOTES GO HERE:

% Some general latex examples and examples making use of the
% macros follow.
%**** IN GENERAL, BE BRIEF. LONG SCRIBE NOTES, NO MATTER HOW WELL WRITTEN,
%**** ARE NEVER READ BY ANYBODY.

\section{课程的内容,意义和特点}
\begin{itemize}
  \item \textit{评价一个算法好坏的标准}: 计算结果的精度(即误差大小)和得到结果需要付出的代价
  \item \textit{误差来源}: 方法误差(也称截断误差或余项)和计算摄入误差
  \item 数值稳定题
  \item 时空复杂性
\end{itemize}
% section 课程的内容,意义和特点 (end)

\section{误差的基本概念}
\subsection{误差和有效数字}
\begin{definition}
  (绝对误差与相对误差) P4
\end{definition}
\begin{definition}
  (绝对误差界与相对误差界) P4
\end{definition}
\begin{definition}
  (有效数字) P4
\end{definition}
\begin{theorem}
  (有效数字与相对误差之间的关系) 粗略地说, 有效数字的位数相当于相对误差的百分数的位数. (P5)
\end{theorem}
% subsection 误差和有效数字 (end)

\subsection{函数求值的误差估计}
(近似函数值$f(a)$或$f(x,y)$的误差界和相对误差界) P5, P6
% subsection 函数求值的误差估计 (end)

\subsection{计算机中数的表示和舍入误差}
\begin{itemize}
  \item 机器数: P8
  \item 阶码
  \item 尾数
  \item 字长
\end{itemize}
% subsection 计算机中数的表示和舍入误差 (end)
% section 误差的基本概念 (end)

\section{数值稳定性和病态问题}
\subsection{算法的稳定性}
\begin{definition}
  (数值稳定与数值不稳定) P9
\end{definition}
% subsection 算法的稳定性 (end)

\subsection{病态数学问题和条件数}
\begin{itemize}
  \item \textit{病态数学问题与良态数学问题}: P10
  \item \textit{条件数}: 通常用条件数的大小来衡量问题的病态程度. (P10)
\end{itemize}
% subsection 病态数学问题和条件数 (end)
% section 数值稳定性和病态问题 (end)
\end{document}
