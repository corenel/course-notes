% !TEX program = xelatex
% !TEX encoding = UTF-8 Unicode

\documentclass[twoside]{article}
\setlength{\oddsidemargin}{0.25 in}
\setlength{\evensidemargin}{-0.25 in}
\setlength{\topmargin}{-0.6 in}
\setlength{\textwidth}{6.5 in}
\setlength{\textheight}{8.5 in}
\setlength{\headsep}{0.75 in}
\setlength{\parindent}{0 in}
\setlength{\parskip}{0.1 in}

%
% ADD PACKAGES here:
%

\usepackage{xeCJK}
\usepackage{amsmath,amsfonts,amsthm,graphicx}
\usepackage{mathtools}
\usepackage{color}   %May be necessary if you want to color links
\usepackage{hyperref}
\usepackage{mleftright}
\hypersetup{
    colorlinks=true, %set true if you want colored links
    linktoc=all,     %set to all if you want both sections and subsections linked
    linkcolor=red,  %choose some color if you want links to stand out
}

%
% The following commands set up the lecnum (lecture number)
% counter and make various numbering schemes work relative
% to the lecture number.
%
\newcounter{lecnum}
\renewcommand{\thepage}{\thelecnum-\arabic{page}}
\renewcommand{\thesection}{\thelecnum.\arabic{section}}
\renewcommand{\theequation}{\thelecnum.\arabic{equation}}
\renewcommand{\thefigure}{\thelecnum.\arabic{figure}}
\renewcommand{\thetable}{\thelecnum.\arabic{table}}

%
% The following macro is used to generate the header.
%
\newcommand{\lecture}[4]{
   \pagestyle{myheadings}
   \thispagestyle{plain}
   \newpage
   \setcounter{lecnum}{#1}
   \setcounter{page}{1}
   \noindent
   \begin{center}
   \framebox{
      \vbox{\vspace{2mm}
    \hbox to 6.28in { {\bf 科学与工程计算方法
        \hfill Winter 2017} }
       \vspace{4mm}
       \hbox to 6.28in { {\Large \hfill Lecture #1: #2  \hfill} }
       \vspace{2mm}
       \hbox to 6.28in { {\it Lecturer: #3 \hfill Scribes: #4} }
      \vspace{2mm}}
   }
   \end{center}
   \markboth{Lecture #1: #2}{Lecture #1: #2}
}
%
% Convention for citations is authors' initials followed by the year.
% For example, to cite a paper by Leighton and Maggs you would type
% \cite{LM89}, and to cite a paper by Strassen you would type \cite{S69}.
% (To avoid bibliography problems, for now we redefine the \cite command.)
% Also commands that create a suitable format for the reference list.
\renewcommand{\cite}[1]{[#1]}
\def\beginrefs{\begin{list}%
        {[\arabic{equation}]}{\usecounter{equation}
         \setlength{\leftmargin}{2.0truecm}\setlength{\labelsep}{0.4truecm}%
         \setlength{\labelwidth}{1.6truecm}}}
\def\endrefs{\end{list}}
\def\bibentry#1{\item[\hbox{[#1]}]}

%Use this command for a figure; it puts a figure in wherever you want it.
%usage: \fig{NUMBER}{SPACE-IN-INCHES}{CAPTION}
\newcommand{\fig}[3]{
    \vspace{#2}
    \begin{center}
        Figure \thelecnum.#1:~#3
    \end{center}
}
% Use these for theorems, lemmas, proofs, etc.
\newtheorem{theorem}{定理}[section]
\newtheorem{lemma}{引理}[section]
\newtheorem{proposition}{命题}[section]
\newtheorem{claim}{Claim}[section]
\newtheorem{corollary}{推论}[section]
\newtheorem{definition}{定义}[section]
% \newenvironment{proof}{{\bf Proof:}}{\hfill\rule{2mm}{2mm}}

% **** IF YOU WANT TO DEFINE ADDITIONAL MACROS FOR YOURSELF, PUT THEM HERE:

\newcommand\E{\mathbb{E}}

\begin{document}
%FILL IN THE RIGHT INFO.
%\lecture{**LECTURE-NUMBER**}{**DATE**}{**LECTURER**}{**SCRIBE**}
\lecture{7}{非线性方程和方程组的解法}{Zhitao Liu}{Yusu Pan}
%\footnotetext{These notes are partially based on those of Nigel Mansell.}

% **** YOUR NOTES GO HERE:

% Some general latex examples and examples making use of the
% macros follow.
%**** IN GENERAL, BE BRIEF. LONG SCRIBE NOTES, NO MATTER HOW WELL WRITTEN,
%**** ARE NEVER READ BY ANYBODY.

\section{引言}\label{sec:ying_yan_}
一元方程的搜索法:有根区间,根的隔离区间
% section 引言 (end)

\section{误差的基本概念}\label{sec:wu_cha_de_ji_ben_gai_nian_}
\subsection{一元方程的基本迭代法}\label{sub:yi_yuan_fang_chen_de_ji_ben_die_dai_fa_}
一元方程的基本迭代法:映内,压缩,不动点,不动点迭代法
% subsection 一元方程的基本概念 (end)

\subsection{基本迭代法及其收敛性}
基本迭代法及其收敛性:迭代法的基本收敛性定理,迭代终止准则
% subsection 基本迭代法及其收敛性 (end)

\subsection{局部收敛性和收敛阶}
% subsection 局部收敛性和收敛阶 (end)

\subsection{收敛性的改善}\label{sub:label_name}
Steffensen迭代法
% subsection 收敛性的改善 (end)
% section 误差的基本概念 (end)

\section{一元方程牛顿迭代法}
\subsection{牛顿迭代法及其收敛性}
牛顿迭代法
\begin{equation*}
  x_{k+1} = x_k - \frac{f(x_k)}{f'(x_k)}
\end{equation*}
% subsection 牛顿迭代法及其收敛性 (end)

\subsection{重根时的牛顿迭代改善}
方法1 (构造$\mu(x)$使得$x^*$为$\mu(x)$的单根)
\begin{equation*}
  \begin{aligned}
    \mu(x) &= \frac{f(x)}{f'(x)} \\
    x_{k+1} &= x_k - \frac{\mu(x_k)}{\mu'(x_k)}
  \end{aligned}
\end{equation*}

方法2 (需要已知根的重数$m$, 不实用)
\begin{equation*}
  x_{k+1} = \varphi(x_k) = x_k - m \frac{f(x_k)}{f'(x_k)}
\end{equation*}
% subsection 重根时的牛顿迭代改善 (end)

\subsection{离散牛顿法}
将导数换为$x_k$与$x_{k-1}$的差分公式
\begin{equation*}
  x_{k+1} = x_k - \frac{x_k - x_{k-1}}{f(x_k) - f(x_{k-1})} f(x_k)
\end{equation*}
% subsection 离散牛顿法 (end)
% section 一元方程牛顿迭代法 (end)

\section{非线性方程的解法}\label{sec:fei_xian_xin_fang_chen_de_jie_fa_}
\subsection{不动点迭代法}
\begin{itemize}
  \item 映内, 压缩
  \item 不动点存在定理, 压缩映射定理
  \item 局部收敛
  \item Jacobi矩阵
\end{itemize}
% subsection 不动点迭代法 (end)

\subsection{牛顿迭代法}
\begin{equation*}
  x^{(k+1)} = x^{(k)} - F'(x^{(k)})^{-1} F(x^{(k)})
\end{equation*}

\begin{itemize}
  \item 牛顿方程
  \item 阻尼牛顿法: $F'(x^*)$奇异或病态时可用, 阻尼因子, 阻尼项
\end{itemize}
% subsection 牛顿迭代法 (end)

\subsection{牛顿点迭代法}
用迭代法求解非线性方程组, 特别是非线性方程组时, 初始值的选取至关重要, 初值不仅影响迭代是否收敛, 而且当方程多解时, 不同的初值可能收敛到不同的解.

牛顿迭代函数
% subsection 牛顿点迭代法 (end)

\subsection{拟牛顿法}
% subsection 拟牛顿法 (end)
% section 非线性方程的解法 (end)
\end{document}
