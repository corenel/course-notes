% !TEX program = xelatex
% !TEX encoding = UTF-8 Unicode

\documentclass[twoside]{article}
\setlength{\oddsidemargin}{0.25 in}
\setlength{\evensidemargin}{-0.25 in}
\setlength{\topmargin}{-0.6 in}
\setlength{\textwidth}{6.5 in}
\setlength{\textheight}{8.5 in}
\setlength{\headsep}{0.75 in}
\setlength{\parindent}{0 in}
\setlength{\parskip}{0.1 in}

%
% ADD PACKAGES here:
%

\usepackage{xeCJK}
\usepackage{amsmath,amsfonts,amsthm,graphicx}
\usepackage{mathtools}
\usepackage{color}   %May be necessary if you want to color links
\usepackage{hyperref}
\usepackage{mleftright}
\hypersetup{
    colorlinks=true, %set true if you want colored links
    linktoc=all,     %set to all if you want both sections and subsections linked
    linkcolor=red,  %choose some color if you want links to stand out
}

%
% The following commands set up the lecnum (lecture number)
% counter and make various numbering schemes work relative
% to the lecture number.
%
\newcounter{lecnum}
\renewcommand{\thepage}{\thelecnum-\arabic{page}}
\renewcommand{\thesection}{\thelecnum.\arabic{section}}
\renewcommand{\theequation}{\thelecnum.\arabic{equation}}
\renewcommand{\thefigure}{\thelecnum.\arabic{figure}}
\renewcommand{\thetable}{\thelecnum.\arabic{table}}

%
% The following macro is used to generate the header.
%
\newcommand{\lecture}[4]{
   \pagestyle{myheadings}
   \thispagestyle{plain}
   \newpage
   \setcounter{lecnum}{#1}
   \setcounter{page}{1}
   \noindent
   \begin{center}
   \framebox{
      \vbox{\vspace{2mm}
    \hbox to 6.28in { {\bf 科学与工程计算方法
        \hfill Winter 2017} }
       \vspace{4mm}
       \hbox to 6.28in { {\Large \hfill Lecture #1: #2  \hfill} }
       \vspace{2mm}
       \hbox to 6.28in { {\it Lecturer: #3 \hfill Scribes: #4} }
      \vspace{2mm}}
   }
   \end{center}
   \markboth{Lecture #1: #2}{Lecture #1: #2}
}
% Use these for theorems, lemmas, proofs, etc.
\newtheorem{theorem}{定理}[section]
\newtheorem{lemma}{引理}[section]
\newtheorem{proposition}{命题}[section]
\newtheorem{claim}{Claim}[section]
\newtheorem{corollary}{推论}[section]
\newtheorem{definition}{定义}[section]
\newtheorem{eg}{例}[section]
% \newenvironment{proof}{{\bf Proof:}}{\hfill\rule{2mm}{2mm}}
%
% **** IF YOU WANT TO DEFINE ADDITIONAL MACROS FOR YOURSELF, PUT THEM HERE:
%
\begin{document}
%FILL IN THE RIGHT INFO.
%\lecture{**LECTURE-NUMBER**}{**DATE**}{**LECTURER**}{**SCRIBE**}
\lecture{7}{非线性方程和方程组的解法}{Zhitao Liu}{Yusu Pan}
%\footnotetext{These notes are partially based on those of Nigel Mansell.}

% **** YOUR NOTES GO HERE:

% Some general latex examples and examples making use of the
% macros follow.
%**** IN GENERAL, BE BRIEF. LONG SCRIBE NOTES, NO MATTER HOW WELL WRITTEN,
%**** ARE NEVER READ BY ANYBODY.

\section{引言}\label{sec:ying_yan_}
\subsection{问题的背景和内容摘要}
\begin{itemize}
  \item 一元非线性方程$f(x)=0$
  \item 多元非线性方程组$f_i(x_1,x_2,\ldots,x_n)=0, i=1,2,\ldots,n$
\end{itemize}
% subsection 问题的背景和内容摘要 (end)
\subsection{一元方程的搜索法}
\begin{itemize}
  \item 有根区间, 根的隔离区间
  \item 搜索法
  \item $m$重根, $m$为奇数时, $f(x)$在点$x^*$处变号; $m$为偶数时不变号
\end{itemize}
\begin{eg}
  (使用搜索法寻找长度为0.2的根的隔离区间)
\end{eg}
% subsection 一元方程的搜索法 (end)
% section 引言 (end)

\section{误差的基本概念}\label{sec:wu_cha_de_ji_ben_gai_nian_}
\subsection{基本迭代法及其收敛性}
\paragraph{一元方程的基本迭代法}
\begin{itemize}
  \item 基本迭代法/不动点迭代法, 单步法的一种
    \begin{equation}
      f(x)=0 \to x=\varphi(x) \to x_{k+1}=\varphi(x_k)
    \end{equation}
  \item 迭代函数$\varphi(x_k)$, 不动点$x^*$
\end{itemize}
% paragraph 一元方程的基本迭代法 (end)
\begin{theorem}
  (迭代法的基本收敛性定理/全局性收敛定理)
  \begin{itemize}
    \item 映内性
      \begin{equation}
        a\le \varphi(x) \le b, \forall x\in[a,b]
      \end{equation}
    \item 压缩性
      \begin{equation}
        |\varphi'(x)|\le L < 1, \forall x\in[a,b]
      \end{equation}
    \item 适定性
    \item 迭代终止准则
      \begin{equation}
        \frac{|x_k-x_{k-1}|}{|x_k|+1}< \epsilon
      \end{equation}
  \end{itemize}
\end{theorem}
\begin{eg}
  (讨论迭代法的收敛性)
\end{eg}
% subsection 基本迭代法及其收敛性 (end)

\subsection{局部收敛性和收敛阶}
\begin{definition}
  (局部收敛)
\end{definition}
\begin{theorem}
  (局部收敛性定理/充分条件)
  \begin{equation}
    |\varphi'(x)|<1
  \end{equation}
\end{theorem}
\begin{definition}
  (收敛阶)
  \begin{equation}
    \begin{aligned}
      \lim_{k\to \infty} \frac{e_{k+1}}{e^p_k}=c \\
      e_k=x_k - x^*
    \end{aligned}
  \end{equation}
  \begin{itemize}
    \item $p$阶收敛
    \item $p=1$, 线性收敛
    \item $p>1$, 超线性收敛
    \item $p=2$, 二次收敛/平方收敛
  \end{itemize}
\end{definition}
\begin{lemma}
  (线性收敛)
  \begin{equation}
    0<|\varphi'(x)|<1
  \end{equation}
\end{lemma}
\begin{eg}
  (线性收敛速度很慢)
\end{eg}
\begin{theorem}
  (整数阶超线性收敛/$p$阶收敛)
  \begin{equation}
    \begin{aligned}
      \varphi^{(l)}(x^*)=0, l=1,2,\ldots,p-1 \\
      \varphi^{(p)}(x^*)\ne0
    \end{aligned}
  \end{equation}
\end{theorem}
\begin{eg}
  (平方收敛比线性收敛快得多)
\end{eg}
% subsection 局部收敛性和收敛阶 (end)

\subsection{收敛性的改善}\label{sub:label_name}
\paragraph{Steffensen迭代法}
\begin{equation}
  \begin{aligned}
    y_k &= \varphi(x_k)\\
    z_k &= \varphi(y_k)\\
    x_{k+1} &= x_k - \frac{{(y_k-x_k)}^2}{z_k-2y_k+x_k}
  \end{aligned}
\end{equation}
% paragraph Steffensen迭代法 (end)
\begin{eg}
  (将迭代函数改造为Steffensen迭代法)
\end{eg}
\begin{theorem}
  \begin{itemize}
    \item 只要$\varphi'(x)\ne1$, 不管原迭代法是否收敛, 由其构成的Stefensen方法至少平方收敛
    \item 若原迭代法的收敛阶已经大于等于2, 则可不必使用Stefensen迭代法
  \end{itemize}
\end{theorem}
% subsection 收敛性的改善 (end)
\begin{eg}
  (用发散的迭代函数构造Steffensen迭代法)
\end{eg}
% section 误差的基本概念 (end)

\section{一元方程牛顿迭代法}
\subsection{牛顿迭代法及其收敛性}
牛顿迭代法
\paragraph{牛顿迭代法/切线法}
\begin{equation}
  x_{k+1} = x_k - \frac{f(x_k)}{f'(x_k)}
\end{equation}
% paragraph 牛顿迭代法/切线法 (end)
\begin{eg}
  (用牛顿迭代法求解)
\end{eg}
\begin{theorem}
  (牛顿法至少二次收敛)
\end{theorem}
\begin{eg}
  (对于某些非线性方程, 牛顿法具有全局收敛性)
\end{eg}
% subsection 牛顿迭代法及其收敛性 (end)

\subsection{重根时的牛顿迭代改善}
\begin{itemize}
  \item 方法1 (构造$\mu(x)$使得$x^*$为$\mu(x)$的单根)
    \begin{equation}
      \begin{aligned}
        \mu(x) &= \frac{f(x)}{f'(x)} \\
        x_{k+1} &= x_k - \frac{\mu(x_k)}{\mu'(x_k)}
      \end{aligned}
    \end{equation}
  \item 方法2
    \begin{equation}
      x_{k+1} = \varphi(x_k) = x_k - m \frac{f(x_k)}{f'(x_k)}
    \end{equation}
  \item 方法1需要求函数的二阶导数, 并且当所求根为单根时, 不能改善本来已经二次收敛的牛顿法; 方法2需要已知根的重数$m$, 不实用.
  \item 对于实际问题, 往往事先并不知道所求根是否是重根, 需要通过试算来判断, 如当牛顿法收敛很慢时通常为重根.
\end{itemize}

% subsection 重根时的牛顿迭代改善 (end)

\subsection{离散牛顿法/割线法}
当不想每步都计算导数, 或者函数不可导时, 用割线的斜率来代替牛顿法中的切线的斜率, 即将导数换为$x_k$与$x_{k-1}$的差分公式
\begin{equation}
  x_{k+1} = x_k - \frac{x_k - x_{k-1}}{f(x_k) - f(x_{k-1})} f(x_k)
\end{equation}
\begin{itemize}
  \item 需要两个初始值$x_0$和$x_1$, 应尽量取在方程$f(x)=0$的根$x^*$附近. 
\end{itemize}
\begin{eg}
  (用离散牛顿法求解)
\end{eg}
% subsection 离散牛顿法 (end)
% section 一元方程牛顿迭代法 (end)

\section{非线性方程的解法}\label{sec:fei_xian_xin_fang_chen_de_jie_fa_}
\begin{equation}
  \begin{aligned}
    F(x)&=0 \\
    x&={ [x_1, \ldots, x_n] }^T\\
    F(x)&={ [f_1(x), \ldots, f_n(x)] }^T
  \end{aligned}
\end{equation}
\subsection{不动点迭代法}
\begin{itemize}
  \item 不动点迭代法
    \begin{equation}
      x_{k+1}=\Phi(x_k)
    \end{equation}
  \item Jacobi矩阵
\end{itemize}
\begin{definition}
  (映内, 压缩, 压缩系数$L$)
\end{definition}
\begin{theorem}
  (Brouwer不动点存在定理)
\end{theorem}
\begin{theorem}
  (压缩映射原理)
\end{theorem}
\begin{definition}
  (局部收敛)
\end{definition}
\begin{theorem}
  (局部收敛的充分条件1/Jacobi矩阵)
\end{theorem}
\begin{theorem}
  (局部收敛的充分条件2)
\end{theorem}
\begin{lemma}
  (局部收敛的充分条件3)
  \begin{itemize}
    \item 若$\Phi'(x)$存在, 且$\|\Phi'(x^*)\|<1$, 则迭代法在$x^*$处局部收敛
    \item 若有$x^*$的邻域$S\of D$, $\Phi'(x)$存在, 且$\|\Phi'(x)\|<1$, 则迭代法在$x^*$处局部收敛
  \end{itemize}
\end{lemma}
\begin{eg}
  (判断是否局部收敛)
\end{eg}
% subsection 不动点迭代法 (end)
\subsection{牛顿迭代法}
\begin{definition}
  (收敛阶)
\end{definition}
\begin{equation}
  x^{(k+1)} = x^{(k)} - F'(x^{(k)})^{-1} F(x^{(k)})
\end{equation}
% subsection 牛顿迭代法 (end)

\subsection{牛顿点迭代法}
\paragraph{牛顿点迭代法}
  \begin{equation}
    x^{(k+1)}=\Phi(x^{(k)}), \Phi(x)=x - {F'(x)}^{-1} F(x)
  \end{equation}
% paragraph 牛顿点迭代法 (end)
\begin{theorem}
  (牛顿迭代法的局部收敛性)
\end{theorem}
\begin{itemize}
  \item 牛顿方程
  \item 阻尼牛顿法: $F'(x^*)$奇异或病态时可用, 阻尼因子, 阻尼项
  \item 用迭代法求解非线性方程组, 特别是非线性方程组时, 初始值的选取至关重要, 初值不仅影响迭代是否收敛, 而且当方程多解时, 不同的初值可能收敛到不同的解.
\end{itemize}
\begin{eg}
  (用牛顿法与阻尼牛顿法求解)
\end{eg}
\begin{eg}
  (初值对牛顿法的影响)
\end{eg}
% subsection 牛顿点迭代法 (end)

\subsection{拟牛顿法}
\subsubsection{Broyden秩1方法}
% subsubsection Broyden秩1方法 (end)
\subsubsection{对称拟牛顿法}
% subsubsection 对称拟牛顿法 (end)
% subsection 拟牛顿法 (end)
% section 非线性方程的解法 (end)
\end{document}
