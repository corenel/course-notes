% !TEX program = xelatex
% !TEX encoding = UTF-8 Unicode

\documentclass[twoside]{article}
\setlength{\oddsidemargin}{0.25 in}
\setlength{\evensidemargin}{-0.25 in}
\setlength{\topmargin}{-0.6 in}
\setlength{\textwidth}{6.5 in}
\setlength{\textheight}{8.5 in}
\setlength{\headsep}{0.75 in}
\setlength{\parindent}{0 in}
\setlength{\parskip}{0.1 in}

%
% ADD PACKAGES here:
%

\usepackage{xeCJK}
\usepackage{amsmath,amsfonts,amsthm,graphicx}
\usepackage{mathtools}
\usepackage{color}   %May be necessary if you want to color links
\usepackage{hyperref}
\usepackage{booktabs}
\usepackage{mleftright}
\hypersetup{
    colorlinks=true, %set true if you want colored links
    linktoc=all,     %set to all if you want both sections and subsections linked
    linkcolor=red,  %choose some color if you want links to stand out
}

%
% The following commands set up the lecnum (lecture number)
% counter and make various numbering schemes work relative
% to the lecture number.
%
\newcounter{lecnum}
\renewcommand{\thepage}{\thelecnum-\arabic{page}}
\renewcommand{\thesection}{\thelecnum.\arabic{section}}
\renewcommand{\theequation}{\thelecnum.\arabic{equation}}
\renewcommand{\thefigure}{\thelecnum.\arabic{figure}}
\renewcommand{\thetable}{\thelecnum.\arabic{table}}

%
% The following macro is used to generate the header.
%
\newcommand{\lecture}[4]{
   \pagestyle{myheadings}
   \thispagestyle{plain}
   \newpage
   \setcounter{lecnum}{#1}
   \setcounter{page}{1}
   \noindent
   \begin{center}
   \framebox{
      \vbox{\vspace{2mm}
    \hbox to 6.28in { {\bf 科学与工程计算方法
        \hfill Winter 2017} }
       \vspace{4mm}
       \hbox to 6.28in { {\Large \hfill Lecture #1: #2  \hfill} }
       \vspace{2mm}
       \hbox to 6.28in { \textit{Lecturer: #3 \hfill Scribes: #4} }
      \vspace{2mm}}
   }
   \end{center}
   \markboth{Lecture #1: #2}{Lecture #1: #2}
}
%
% Convention for citations is authors' initials followed by the year.
% For example, to cite a paper by Leighton and Maggs you would type
% \cite{LM89}, and to cite a paper by Strassen you would type \cite{S69}.
% (To avoid bibliography problems, for now we redefine the \cite command.)
% Also commands that create a suitable format for the reference list.
\renewcommand{\cite}[1]{[#1]}
\def\beginrefs{\begin{list}%
        {[\arabic{equation}]}{\usecounter{equation}
         \setlength{\leftmargin}{2.0truecm}\setlength{\labelsep}{0.4truecm}%
         \setlength{\labelwidth}{1.6truecm}}}
\def\endrefs{\end{list}}
\def\bibentry#1{\item[\hbox{[#1]}]}

%Use this command for a figure; it puts a figure in wherever you want it.
%usage: \fig{NUMBER}{SPACE-IN-INCHES}{CAPTION}
\newcommand{\fig}[3]{
                        \vspace{#2}
                        \begin{center}
                        Figure \thelecnum.#1:~#3
                        \end{center}
        }
% Use these for theorems, lemmas, proofs, etc.
\newtheorem{theorem}{定理}[section]
\newtheorem{lemma}{引理}[section]
\newtheorem{proposition}{命题}[section]
\newtheorem{claim}{Claim}[section]
\newtheorem{corollary}{推论}[section]
\newtheorem{definition}{定义}[section]
\newtheorem{eg}{例}[section]
% \newenvironment{proof}{{\bf Proof:}}{\hfill\rule{2mm}{2mm}}

% **** IF YOU WANT TO DEFINE ADDITIONAL MACROS FOR YOURSELF, PUT THEM HERE:

\begin{document}
%FILL IN THE RIGHT INFO.
%\lecture{**LECTURE-NUMBER**}{**DATE**}{**LECTURER**}{**SCRIBE**}
\lecture{6}{常微分方程的数值解法}{Zhitao Liu}{Yusu Pan}
%\footnotetext{These notes are partially based on those of Nigel Mansell.}

% **** YOUR NOTES GO HERE:

% Some general latex examples and examples making use of the
% macros follow.
%**** IN GENERAL, BE BRIEF. LONG SCRIBE NOTES, NO MATTER HOW WELL WRITTEN,
%**** ARE NEVER READ BY ANYBODY.

\section{引言}
\begin{itemize}
  \item 常微分方程初值问题的一般形式
    \begin{equation}
        \frac{dy}{dx}=f(x,y)
    \end{equation}
    \begin{equation}
          y(x_0)=y_0
    \end{equation}
  \item 两点边值问题 (第一类边界条件)的一般形式
    \begin{equation}
      y''=f(x,y,y')
    \end{equation}
    \begin{equation}
      y(a)=\alpha, y(b)=\beta
    \end{equation}
\end{itemize}

\section{初值问题的数值解法}
\subsection{Euler方法及其截断误差}
\subsubsection{Euler公式}
\begin{itemize}
  \item 显式Euler公式
    \begin{equation}
        y_{n+1}=y_n + hf(x_n, y_n)
    \end{equation}
  \item 隐式Euler公式/后退Euler公式
    \begin{equation}
      y_{n+1}=y_n + hf(x_{n+1},y_{n+1})
    \end{equation}
  \item 梯形公式
    \begin{equation}
      y_{n+1} = y_n + \frac{h}{2} [f(x_n,y_n), f(x_{n+1},y_{n+1})]
    \end{equation}
  \item 差分公式: 统称Euler公式与梯形公式
  \item 单步法: 由$y_n$去计算$y_{n+1}$
\end{itemize}
% subsubsection Euler公式 (end)
\subsubsection{隐式公式的计算}
\begin{itemize}
  \item 如果$f(x,y)$是$y$的线性函数, 则隐式公式可以显式计算
  \item 如果$f(x,y)$是$y$的非线性函数, 则可以用迭代法求解
  \item 简单迭代公式
    \begin{equation}
      \begin{cases}{}
        y^{(0)}_{n+1} = y_n + hf(x_n,y_n)\\
        y^{(k+1)}_{n+1} = y_n + \frac{h}{2} [f(x_n, y_n)+f(x_{n+1},y^{(k)}_{n+1})], &k=0,1,2,\ldots
      \end{cases}
    \end{equation}
  \item 预测-校正技术: 先用显式方法计算, 预测一个值$y^p_{n+1}$, 为隐式公式提供一个好的迭代初值, 然后隐式公式迭代一次, 得到$y_{n+1}$
    \begin{equation}
      \begin{cases}{}
        y^{(0)}_{n+1} = y_n + hf(x_n,y_n)\\
        y^{(k+1)}_{n+1} = y_n + \frac{h}{2} [f(x_n, y_n)+f(x_{n+1},y^p_{n+1})]
      \end{cases}
    \end{equation}
  \item 改进Euler公式 (与预测-校正技术等价的显式公式)
    \begin{equation}
      y_{n+1}=y_n + \frac{h}{2} [f(x_n,y_n) + f(x_{n+1}, y_n + h f(x_n, y_n))]
    \end{equation}
\end{itemize}
\begin{eg}
  (用显式Euler法, 隐式Euler法和梯形公式求解微分初值问题)
\end{eg}
% subsubsection 隐式公式的计算 (end)
\subsubsection{局部截断误差和方法的阶}
\paragraph{整体截断误差}
从$x_0$开始计算,如果考虑每一步产生的误差,直到$x_n$,则有误差$e(n)=y(x_{n+1})-y_n$。
% paragraph 整体截断误差 (end)
\begin{definition}
  单步法的局部截断误差
  \begin{equation}
    T_{n+1}=y(x_{n+1})-y(x_n)-\phi(x_n, x_{n+1}, y(x_n),y(x_{n+1}),h)
  \end{equation}
\end{definition}
\begin{definition}
  如果给定方法的局部截断误差是
  \begin{equation}
    T_{n+1}=O(h^{p+1})
  \end{equation}
  其中$p\ge 1$为整数,则称是$p$阶的,或具有$p$阶精度。
\end{definition}
\begin{definition}
  局部截断误差主项$g(x_n, y(x_n)) h^{p+1}$
  \begin{equation}
      T_{n+1}=g(x_n, y(x_n)) h^{p+1} + O(h^{p+2})
  \end{equation}
\end{definition}

显式与隐式Euler公式均为一阶方法。梯形公式是两者的算术平均,其局部截断误差也是两者的算术平均,为二阶精度。
% subsubsection 局部截断误差和方法的阶 (end)
% subsection Euler方法及其截断误差 (end)
\subsection{Runge-Kutta法}
        方法的阶越高,方法就越精确,这启发人们用区间上若干个点的导数$f$,将其线性组合得到平均斜率,以提高方法的阶,这就是Runge-Kutta方法的基本思想。

\begin{equation}
    \begin{aligned}
        y_{n+1} &= y_n + h \sum_{i=1}^{L} \lambda_i k_i \\
        k_i &= f(x_n + c_i h, y_n + c_i h \sum_{j=1}{i-1} a_{ij} k_j), i=2,3\dots L \\
    \end{aligned}
\end{equation}

\begin{itemize}
  \item $L$级$p$阶Runge-Kutta方法
  \item 经典R-K方法(4级4阶)
    \begin{equation}
      \begin{aligned}
        y_{n+1} &= y_n + \frac{h}{6} (k_1 + 2k_2 + 2k_3 + k_4) \\
        k1 &=f(x_n, y_n) \\
        k2 &=f(x_n + \frac{h}{2}, y_n + \frac{h}{2} k_1) \\
        k3 &=f(x_n + \frac{h}{2}, y_n + \frac{h}{2} k_2) \\
        k4 &=f(x_n + h, y_n + h k_3) \\
      \end{aligned}
    \end{equation}
  \item R-K-Fehlberg方法(由一个5级4阶R-K公式和一个6级5阶R-K公式组成)
  \item 隐式R-K法:
    \begin{itemize}
      \item 1级2阶中心公式
      \item 2级2阶梯形公式
      \item 2级4阶R-K公式
    \end{itemize}
\end{itemize}
\begin{eg}
  (用一阶显式Euler公式, 改进Euler公式和经典R-K公式计算微分方程初值问题)
\end{eg}
% subsection Runge-Kutta法 (end)
\subsection{单步法的稳定性}


\begin{itemize}
  \item 单步法的统一形式, 增量函数$\psi (x_n,y_n,h)$
    \begin{equation}
      y_{n+1} = y(n) + h \psi (x_n,y_n,h)
    \end{equation}
  \item 条件稳定, 无条件稳定 (如梯形公式)
\end{itemize}
\begin{definition}
  (方法收敛的定义)
\end{definition}
\begin{theorem}
  (Lipschitz条件的定义)
\end{theorem}
\begin{lemma}
  满足Lipschitz条件, 初值精确, 则显式Euler法, 改进Euler法和R-K方法是收敛的.
\end{lemma}
\begin{lemma}
  一个方法的整体截断误差比局部截断误差低一阶, 所以常常通过求出局部截断误差w
\end{lemma}
\begin{definition}
  (绝对稳定, 绝对稳定区域, 绝对稳定区间)
  \begin{equation}
    y_{n+1} = E (\lambda h) y_n
  \end{equation}
\end{definition}
\begin{eg}
  (经典R-K法绝对稳定区间$-2.785\le \lambda h \le 0$求步长$h$)
\end{eg}
% subsection 单步法的稳定性 (end)
\subsection{线性多步法}
\paragraph{线性多步法}
利用前面已经算出的$r+1$个值, 计算$y_{n+1}$. 构造如下的线性多项式,使其具有$p$阶精度
\begin{equation}
    \begin{aligned}
        y_{n+1} &=\alpha_0 y_n + \alpha_1 y_{n-1} + \cdots \alpha_r y_{n-r} + h(\beta_{-1}f_{n+1} + \beta_0 f_n + \cdots + \beta_r f_{n-r}) \\
        &= \sum^r_{k=0} \alpha_k y_{n-k} + h \sum^r_{k=-1} \beta_k f_{n-k}
    \end{aligned}
\end{equation}

记住推导方法,得到以下公式

\begin{equation}
  \begin{cases}
        \sum^r_{k=0} \alpha_k = 1 \\
        \sum^r_{k=1} {(-k)}^j \alpha_k + j \sum^r_{k=-1} { (-k) }^(j-1) \beta_k = 1,& j=1,2,\ldots,p
  \end{cases}
\end{equation}

通过同阶的单步法从$y_0$迭代计算,为线性多步法提供所需的一系列初值$y_1, y_2, \dots, y_r$.
% paragraph 线性多步法 (end)
\begin{itemize}
  \item Adams 4步4阶显式公式 $\star$
  \item Milne 4步4阶显式公式 $\star$
  \item 3步4阶隐式Adams公式
  \item 3步4阶隐式Hamming公式 $\star$
  \item Simpson隐式公式
\end{itemize}
\begin{eg}
  (4步4阶显式Milne公式, 3步4阶隐式Hamming公式计算)
\end{eg}
% subsection 线性多步法 (end)
\subsection{预测-校正技术和外推技巧}
\subsubsection{修正Hamming公式}
    \begin{enumerate}
        \item \textbf{预测}:
        \item \textbf{修正}:
        \item \textbf{校正}:
        \item \textbf{修正}:
    \end{enumerate}

% subsubsection 修正Hamming公式 (end)
\subsubsection{Adams预测-校正公式}
    \begin{enumerate}
        \item \textbf{预测}:
        \item \textbf{修正}:
        \item \textbf{校正}:
        \item \textbf{修正}:
    \end{enumerate}
% subsubsection Adams预测-校正公式 (end)
\begin{eg}
  (Milne-Hamming预测-校正公式和修正Hamming法计算)
\end{eg}
% subsection 预测-校正技术和外推技巧 (end)

\section{一阶常微分方程的数值解法}
\subsection{一阶方程组和高阶方程}
\begin{itemize}
  \item 一阶方程组可化为向量形式
  \item 高阶方程可降为一阶方程组
\end{itemize}
% subsection 一阶方程组和高阶方程 (end)
\subsection{刚性方程 (组)}
考虑线性常系数常微分方程组

\begin{equation}
  \frac{dy}{dx} = \mathbf{Ay} + \mathbf{b}, x\in[a,b]
\end{equation}

通解为
\begin{equation}
  \mathbf{y}(x) = \sum_{k=1}^m c_k e^{\lambda_k x} \varphi _k + \psi(x)
\end{equation}
\begin{itemize}
  \item 瞬态解$\sum_{k=1}^m c_k e^{\lambda_k x} \varphi _k$、稳态解$\psi(x)$、快瞬态解 ($|Re(\lambda_k|$较大的项)、慢瞬态解
  \item 刚性现象: 若一个解中既含快瞬态解又含慢瞬态解, 且解分量之间的数量级相差很大, 则快瞬态解严重影响数值解的稳定性和精度.整个区间上计算均受快瞬态解的制约, 在很长的区间内必须保持小步长
  \item 刚性方程、刚性方程组、刚性比
    \begin{equation}
      s = \frac{\max_{1\le k\le m}{|Re(\lambda_k)|}}{\min_{1\le k\le m}{|Re(\lambda_k)|}} \gg 1
    \end{equation}
\end{itemize}
\begin{eg}
  (用隐式Euler法, 梯形公式, 2级4阶隐式R-K法和4阶经典R-K法求解)
  \begin{itemize}
    \item 在非刚性阶段宁愿用小步长显式方法, 因为当问题是非线性时, 显式方法比隐式方法简单
    \item 在刚性阶段, 宜用高稳定的隐式方法, 因为它可以放大步长, 特别是A稳定的方法
  \end{itemize}
\end{eg}
% subsection 刚性方程 (组) (end)
\subsection{刚性方程 (组)的数值方法}
\begin{definition}
  (A稳定)
\end{definition}
\begin{itemize}
    \item 隐式Euler法, 梯形公式, 2级4阶隐式R-K法均为A稳定
    \item 任何显式多步法和显式R-K法都不可能是A稳定的
    \item A稳定的隐式线性多步法的阶不能超过2, 其中具有最小误差常数的公式是梯形公式
\end{itemize}
% subsection 刚性方程 (组)的数值方法 (end)
% section 一阶常微分方程的数值解法 (end)

\section{边值问题的打靶法和差分法}\label{sec:bian_zhi_wen_ti_de_da_ba_fa_he_chai_fen_fa_}
\paragraph{线性两点边值问题}
\begin{equation}
  \left\{
    \begin{aligned}
        & y'' + p(x)y' + q(x)y = f(x) \\
        & y(a) = \alpha \\
        & y(b)= \beta
    \end{aligned}
  \right.
\end{equation}
% paragraph 线性两点边值问题 (end)
\subsection{打靶法}\label{sub:da_ba_fa_}
\subsubsection{线性打靶法}\label{ssub:xian_xing_da_ba_fa_}
  用解析的思想,将线性边值问题转化为两个初值问题,求得两个初值问题的解,即可得到边值问题的解的表达式.
  \begin{equation}
    \begin{cases}{}
      Ly_1=y_1'' + p(x)y_1' + q(x)y_1=f(x),&y_1(a)=a, y_1'(a)=0 \\
      Ly_2=y_2'' + p(x)y_2' + q(x)y_2=0,&y_2(a)=0, y_2'(a)=1 \\
      y(x) = y_1(x) + \frac{\beta - y_1(b)}{y_2(b)} y_2(x)
    \end{cases}
  \end{equation}
% subsubsection 线性打靶法 (end)
\subsubsection{非线性打靶法}\label{ssub:fei_xian_xing_da_ba_fa_}
对于如下的非线性边值问题
\begin{equation}
  \left\{
    \begin{aligned}
      & y''=f(x,y,y') \\
      & y(a)=\alpha, y'(a)=s_k
    \end{aligned}
  \right.
\end{equation}

令$z=y'$,将上述的二阶方程化为一阶方程组

\begin{equation}
  \left\{
    \begin{aligned}
      & y'=z, y(a)=\alpha\\
      & z' = f(x,y,z), z(a)=s_k
    \end{aligned}
  \right.
\end{equation}

主要是斜率$s_k$的选取,可以使用第七章的迭代法 (如割线法)求得.
% subsubsection 非线性打靶法 (end)
% subsection 打靶法 (end)

\subsection{差分法}\label{sub:chai_fen_fa_}
\paragraph{线性两点边值问题差分法}
\begin{equation}
  \begin{cases}{}
    (2h^2q_1-4)y_1 + (2+hp_1)y_2 = 2h^2f_1 - (2-hp_1)\alpha \\
    (2-hp_k)y_{k-1} + (2h^2q_k-4)y_k + (2+hp_k) y_{k+1} = 2h^2f_k, &k=2,3,\ldots,n-2 \\
    (2-hp_{n-1})y_{n-2} + (2h^2q_{n-1} -4)y_{n-1} = 2h^2f_{n-1} - (2+hp_{n-1})\beta
  \end{cases}
\end{equation}
% paragraph 线性两点边值问题差分法 (end)
\begin{eg}
  (用差分法求解线性两点边值问题)
\end{eg}
\paragraph{非线性两点边值问题的差分法}
% paragraph 非线性两点边值问题的差分法 (end)
% subsection 差分法 (end)
% section 边值问题的打靶法和差分法 (end)

\section{小结}\label{sec:xiao_jie_}
  \begin{itemize}
    \item 对于一个常微分方程 (组), 首先判别其性态是刚性还是非刚性的. 对于非刚性问题, 用显式方法计算简单.
    \item 对于刚性问题, 在非刚性阶段, 宜选用小步长经典R-K法; 在刚性阶段, 选用梯形公式等隐式R-K法. 其中特别对非线性刚性问题, 常用的是A稳定的隐式方法.
    \item 在假定边值问题解存在唯一的前提下, 可以采用打靶法和差分法.
  \end{itemize}
% section 小结 (end)
\end{document}
