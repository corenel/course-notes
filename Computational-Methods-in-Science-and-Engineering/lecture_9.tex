% !TEX program = xelatex
% !TEX encoding = UTF-8 Unicode

\documentclass[twoside]{article}
\setlength{\oddsidemargin}{0.25 in}
\setlength{\evensidemargin}{-0.25 in}
\setlength{\topmargin}{-0.6 in}
\setlength{\textwidth}{6.5 in}
\setlength{\textheight}{8.5 in}
\setlength{\headsep}{0.75 in}
\setlength{\parindent}{0 in}
\setlength{\parskip}{0.1 in}

%
% ADD PACKAGES here:
%

\usepackage{xeCJK}
\usepackage{amsmath,amsfonts,amsthm,amssymb,graphicx}
\usepackage{mathtools}
\usepackage{color}   %May be necessary if you want to color links
\usepackage{hyperref}
\usepackage{mleftright}
\hypersetup{
    colorlinks=true, %set true if you want colored links
    linktoc=all,     %set to all if you want both sections and subsections linked
    linkcolor=red,  %choose some color if you want links to stand out
}

%
% The following commands set up the lecnum (lecture number)
% counter and make various numbering schemes work relative
% to the lecture number.
%
\newcounter{lecnum}
\renewcommand{\thepage}{\thelecnum-\arabic{page}}
\renewcommand{\thesection}{\thelecnum.\arabic{section}}
\renewcommand{\theequation}{\thelecnum.\arabic{equation}}
\renewcommand{\thefigure}{\thelecnum.\arabic{figure}}
\renewcommand{\thetable}{\thelecnum.\arabic{table}}

%
% The following macro is used to generate the header.
%
\newcommand{\lecture}[4]{
   \pagestyle{myheadings}
   \thispagestyle{plain}
   \newpage
   \setcounter{lecnum}{#1}
   \setcounter{page}{1}
   \noindent
   \begin{center}
   \framebox{
      \vbox{\vspace{2mm}
    \hbox to 6.28in { {\bf 科学与工程计算方法
        \hfill Winter 2017} }
       \vspace{4mm}
       \hbox to 6.28in { {\Large \hfill Lecture #1: #2  \hfill} }
       \vspace{2mm}
       \hbox to 6.28in { {\it Lecturer: #3 \hfill Scribes: #4} }
      \vspace{2mm}}
   }
   \end{center}
   \markboth{Lecture #1: #2}{Lecture #1: #2}
}
%
% Convention for citations is authors' initials followed by the year.
% For example, to cite a paper by Leighton and Maggs you would type
% \cite{LM89}, and to cite a paper by Strassen you would type \cite{S69}.
% (To avoid bibliography problems, for now we redefine the \cite command.)
% Also commands that create a suitable format for the reference list.
\renewcommand{\cite}[1]{[#1]}
\def\beginrefs{\begin{list}%
        {[\arabic{equation}]}{\usecounter{equation}
         \setlength{\leftmargin}{2.0truecm}\setlength{\labelsep}{0.4truecm}%
         \setlength{\labelwidth}{1.6truecm}}}
\def\endrefs{\end{list}}
\def\bibentry#1{\item[\hbox{[#1]}]}

%Use this command for a figure; it puts a figure in wherever you want it.
%usage: \fig{NUMBER}{SPACE-IN-INCHES}{CAPTION}
\newcommand{\fig}[3]{
    \vspace{#2}
    \begin{center}
        Figure \thelecnum.#1:~#3
    \end{center}
}
% Use these for theorems, lemmas, proofs, etc.
\newtheorem{theorem}{定理}[section]
\newtheorem{algo}{算法}[section]
\newtheorem{lemma}{引理}[section]
\newtheorem{proposition}{命题}[section]
\newtheorem{claim}{Claim}[section]
\newtheorem{corollary}{推论}[section]
\newtheorem{definition}{定义}[section]
\newtheorem{eg}{例}[section]
% \newenvironment{proof}{{\bf Proof:}}{\hfill\rule{2mm}{2mm}}

% **** IF YOU WANT TO DEFINE ADDITIONAL MACROS FOR YOURSELF, PUT THEM HERE:

\begin{document}
%FILL IN THE RIGHT INFO.
%\lecture{**LECTURE-NUMBER**}{**DATE**}{**LECTURER**}{**SCRIBE**}
\lecture{9}{矩阵特征值问题的数值解法}{Zhitao Liu}{Yusu Pan}
%\footnotetext{These notes are partially based on those of Nigel Mansell.}

% **** YOUR NOTES GO HERE:

% Some general latex examples and examples making use of the
% macros follow.
%**** IN GENERAL, BE BRIEF. LONG SCRIBE NOTES, NO MATTER HOW WELL WRITTEN,
%**** ARE NEVER READ BY ANYBODY.

\section{引言}
\subsection{问题的背景和内容概要}
\begin{itemize}
  \item 特征方程, 特征值, 特征向量
  \item 本章介绍两类实用迭代法--幂法 (及其变形)和QR算法. 前者主要用于求部分特征值和特征向量, 后者用于求全部特征值和特征向量.
\end{itemize}
% subsection 问题的背景和内容概要 (end)
\subsection{特征值的扰动和条件数}
\begin{theorem}
  (扰动后的矩阵特征值误差)
\end{theorem}
\begin{itemize}
  \item 矩阵特征值的条件数
  \item 病态矩阵
\end{itemize}
% subsection 特征值的扰动和条件数 (end)
% section 引言 (end)

\section{幂法及其变形}
\subsection{幂法和外推加速}
\begin{itemize}
  \item 主特征值 (模最大), 主特征向量
  \item 幂法的原始形式
    \begin{equation}
      \begin{aligned}
        v_k &= A v_{k-1} = A^k v_0 \\
            &\approx \lambda^k \alpha_1 x_1, \frac{{ (v_{k+1}) }_l}{{ (v_k) }_l}\approx \lambda_1
      \end{aligned}
    \end{equation}
\end{itemize}
\begin{algo}
  (幂法)
  \begin{itemize}
    \item 取$u_0 = [1,\ldots,1]^T$
    \item 迭代过程
      \begin{enumerate}
        \item $v_k = A u_{k-1}$
        \item $m_k=\max{(v_k)}$
        \item $u_k=v_k/m_k$
        \item 当$\frac{|m_k-m_{k-1}|}{1+|m_k|}<\epsilon$时终止, $u_k$为$x_1$, $m_k$为$\lambda$
      \end{enumerate}
  \end{itemize}
\end{algo}
\begin{theorem}
  (幂法的收敛性1)
\end{theorem}
\begin{eg}
  (用幂法求主特征值和主特征向量)
\end{eg}
\begin{theorem}
  (幂法的收敛性2)
\end{theorem}
\begin{algo}
  (外推加速的幂法/Aitken外推法)
  \begin{itemize}
    \item 取$u_0 = [1,\ldots,1]^T$
    \item 迭代过程
      \begin{enumerate}
        \item $v_k = A u_{k-1}$
        \item $m_k=\max{(v_k)}$
          \begin{equation}
            \tilde{m}_k=m_{k-2} - \frac{{(m_{k-1}-m_{k-2})}^2}{m_k-2m_{k-1}+m_{k-2}}
          \end{equation}
        \item $u_k=v_k/m_k$
        \item 当$\frac{|\tilde{m}_k-\tilde{m}_{k-1}|}{1+|\tilde{m}_k|}<\epsilon$时终止, 并计算$\tilde{u}_k$
          \begin{equation}
            { (\tilde{u}_k) }_j =
            \begin{cases}{}
              1, &{(u_k)}_j=1 \\
              { (u_{k-2}) }_j - \frac{{ [{ (u_{k-1}) }_j-{ (u_{k-2}) }_j] }^2}{{ (u_k) }_j-2{ (u_{k-1}) }_j+{ (u_{k-2}) }_j}, &{(u_k)}_j\ne1 \\
            \end{cases}
          \end{equation}
      \end{enumerate}
  \end{itemize}
\end{algo}
% subsection 幂法和外推加速 (end)
\subsection{反幂法和原点位移}
\begin{itemize}
  \item 反幂法: 求$A$的最小特征值及特征向量, 只需对$A^{-1}$使用幂法
    \begin{equation}
      \begin{cases}{}
        v_k = A^{-1}u_{k-1} \\
        m_k = \max{(v_k)} \\
        u_k = v_k / m_k
      \end{cases}
    \end{equation}
  \item 原点位移的反幂法: 求$A$距离点$s$严格最近的特征值
    \begin{equation}
      \begin{cases}{}
        v_k = {(A-sI)}^{-1}u_{k-1} \\
        m_k = \max{(v_k)} \\
        u_k = v_k / m_k
      \end{cases}
    \end{equation}
  \item 动态原点位移: 使用原点位移的反幂法对幂法加速
\end{itemize}
\begin{eg}
  (用反幂法和原点位移的反幂法求解)
\end{eg}
\begin{eg}
  (动态原点位移)
\end{eg}
% subsection 反幂法和原点位移 (end)
\subsection{对称矩阵的修正幂法}
\begin{algo}
  (实对称矩阵的修正幂法)
  \begin{itemize}
    \item 取$u_0 = \frac{1}{\sqrt{n}} {[1,\ldots,1]}^T$
    \item 迭代过程
      \begin{enumerate}
        \item $v_k = A u_{k-1}$
        \item $m_k=v_k^T u_{k-1}$
        \item $u_k=v_k/{ \|v_k\| }_2$
        \item 当$\frac{|m_k-m_{k-1}|}{1+|m_k|}<\epsilon$时终止
      \end{enumerate}
  \end{itemize}
\end{algo}
\begin{eg}
  (幂法与修正幂法的对比)
\end{eg}
% subsection 对称矩阵的修正幂法 (end)
% section 幂法及其变形 (end)

\section{矩阵的两种正交变换}
\subsection{平面旋转变换和镜面反射变换}
\begin{definition}
  (平面旋转变换/Givens变换矩阵)
\end{definition}
\begin{theorem}
  (平面旋转矩阵的性质)
\end{theorem}
\begin{eg}
  (构造平面旋转矩阵)
\end{eg}
\begin{definition}
  (初等镜面反射矩阵/Householder变换矩阵)
\end{definition}
\begin{theorem}
  (构造镜面反射矩阵1)
\end{theorem}
\begin{lemma}
  (构造镜面反射矩阵2)
\end{lemma}
\begin{eg}
  (构造镜面反射矩阵)
\end{eg}
\begin{lemma}
  (构造镜面反射矩阵3)
\end{lemma}
\paragraph{矩阵的收缩方法}
% paragraph 矩阵的收缩方法 (end)
% subsection 平面旋转变换和镜面反射变换 (end)
\subsection{化矩阵为Hessenberg形}
\begin{definition}
  (Hessenberg矩阵/可约与不可约)
\end{definition}
\begin{theorem}
  任何实方阵都可以通过正交相似变化化为Henssenberg形
  \begin{equation}
    B=Q^T AQ
  \end{equation}
\end{theorem}
% subsection 化矩阵为Hessenberg形 (end)
\subsection{矩阵的QR分解}
\begin{theorem}
  (QR分解定理$\bigstar$)
  \begin{equation}
    \begin{aligned}
      A=QR \\
      R=A_n, Q=H_1 H_2 \cdots H_{n-1}
    \end{aligned}
  \end{equation}
\end{theorem}
% subsection 矩阵的QR分解 (end)
% section 矩阵的QR分解 (end)

\section{QR算法}
\subsection{QR算法及其收敛性}
\begin{theorem}
  (Schur分块上三角)
\end{theorem}
\begin{algo}
  (基本QR算法)
  \begin{itemize}
    \item 初始化: $A_1$为$A$的Hessenberg形
    \item QR变换
      \begin{enumerate}
        \item $A_k=Q_kR_k$ (QR分解)
        \item $A_{k+1}=Q^T_k A_k Q_k = R_k Q_k$ (正交相似变换)
      \end{enumerate}
  \end{itemize}
\end{algo}
\begin{theorem}
  (基本QR算法的性质)
\end{theorem}
\begin{lemma}
\end{lemma}
\begin{theorem}
  (QR算法的收敛性)
\end{theorem}
\begin{theorem}
  (QR算法求$A$的全部特征值)
\end{theorem}
% subsection QR算法及其收敛性 (end)

\subsection{QR算法的改善}
\begin{itemize}
  \item 划分和收缩
  \item 原点位移
    \begin{algo}
      (原点位移的QR变换)
    \end{algo}
\end{itemize}
% subsection QR算法的改善 (end)

\subsection{双步隐式QR算法*}
\begin{algo}
  (双步隐式QR变换)
\end{algo}
\begin{algo}
  (双步隐式QR算法)
\end{algo}
% subsection 双步隐式QR算法 (end)
% section QR算法 (end)
\end{document}

