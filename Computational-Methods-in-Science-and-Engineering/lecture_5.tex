% !TEX program = xelatex
% !TEX encoding = UTF-8 Unicode

\documentclass[twoside]{article}
\setlength{\oddsidemargin}{0.25 in}
\setlength{\evensidemargin}{-0.25 in}
\setlength{\topmargin}{-0.6 in}
\setlength{\textwidth}{6.5 in}
\setlength{\textheight}{8.5 in}
\setlength{\headsep}{0.75 in}
\setlength{\parindent}{0 in}
\setlength{\parskip}{0.1 in}

%
% ADD PACKAGES here:
%

\usepackage{xeCJK}
\usepackage{amsmath,amsfonts,amsthm,graphicx}
\usepackage{mathtools}
\usepackage{color}   %May be necessary if you want to color links
% \usepackage{hyperref}
\usepackage{enumitem}
\usepackage{booktabs}
\usepackage{mleftright}
% \hypersetup{
    % colorlinks=true, %set true if you want colored links
    % linktoc=all,     %set to all if you want both sections and subsections linked
    % linkcolor=red,  %choose some color if you want links to stand out
% }

%
% The following commands set up the lecnum (lecture number)
% counter and make various numbering schemes work relative
% to the lecture number.
%
\newcounter{lecnum}
\renewcommand{\thepage}{\thelecnum-\arabic{page}}
\renewcommand{\thesection}{\thelecnum.\arabic{section}}
\renewcommand{\theequation}{\thelecnum.\arabic{equation}}
\renewcommand{\thefigure}{\thelecnum.\arabic{figure}}
\renewcommand{\thetable}{\thelecnum.\arabic{table}}

%
% The following macro is used to generate the header.
%
\newcommand{\lecture}[4]{
   \pagestyle{myheadings}
   \thispagestyle{plain}
   \newpage
   \setcounter{lecnum}{#1}
   \setcounter{page}{1}
   \noindent
   \begin{center}
   \framebox{
      \vbox{\vspace{2mm}
    \hbox to 6.28in { {\bf 科学与工程计算方法
        \hfill Winter 2017} }
       \vspace{4mm}
       \hbox to 6.28in { {\Large \hfill Lecture #1: #2  \hfill} }
       \vspace{2mm}
       \hbox to 6.28in { \textit{Lecturer: #3 \hfill Scribes: #4} }
      \vspace{2mm}}
   }
   \end{center}
   \markboth{Lecture #1: #2}{Lecture #1: #2}
}
% Use these for theorems, lemmas, proofs, etc.
\newtheorem{theorem}{定理}[section]
\newtheorem{algo}{算法}[section]
\newtheorem{lemma}{引理}[section]
\newtheorem{proposition}{命题}[section]
\newtheorem{claim}{Claim}[section]
\newtheorem{corollary}{推论}[section]
\newtheorem{definition}{定义}[section]
\newtheorem{eg}{例}[section]
% \newenvironment{proof}{{\bf Proof:}}{\hfill\rule{2mm}{2mm}}

% **** IF YOU WANT TO DEFINE ADDITIONAL MACROS FOR YOURSELF, PUT THEM HERE:
\newcommand{\norm}[1]{\left\lVert#1\right\rVert}

\begin{document}
%FILL IN THE RIGHT INFO.
%\lecture{**LECTURE-NUMBER**}{**DATE**}{**LECTURER**}{**SCRIBE**}
\lecture{5}{数值积分和数值微分}{Zhitao Liu}{Yusu Pan}
%\footnotetext{These notes are partially based on those of Nigel Mansell.}

% **** YOUR NOTES GO HERE:

% Some general latex examples and examples making use of the
% macros follow.  
%**** IN GENERAL, BE BRIEF. LONG SCRIBE NOTES, NO MATTER HOW WELL WRITTEN,
%**** ARE NEVER READ BY ANYBODY.

\section{引言}
\begin{itemize}
  \item \textit{微分方程初值问题}
    \begin{equation}
      \begin{aligned}
        \begin{cases}{}
          \frac{dy}{dx}=f(x,y) \\
          y(x_0)=y_0
        \end{cases}
        \\
        y(x)=y_0 + \int_{x_0}^x f(x,y)dx
      \end{aligned}
    \end{equation}
  \item 求积节点$x_k$, 求积系数$A_k$, 近似求积公式$I_n(f)$
    \begin{equation}
      I_n(f)=\sum^n_{k=0} A_k f(x_k)
    \end{equation}
\end{itemize}
% section 引言 (end)

\section{梯形公式和Simpson求积公式}
\subsection{梯形公式和Simpson公式}
\paragraph{梯形公式}
(用1次Lagrange插值多项式近似代替被积函数$f(x)$)
\begin{equation}
  \begin{aligned}
    T_1 &= \frac{b-a}{2} (f(a) + f(b))
    R_{T_1} (f) = -\frac{h^3}{12}f''(\eta), \eta\in(a,b)
  \end{aligned}
\end{equation}
% paragraph 梯形公式 (end)
\paragraph{Simpson求积公式}
(用2次Lagrange插值多项式近似代替被积函数$f(x)$)
\begin{equation}
  S_1 = \int^b_a L_2(x)dx = \frac{h}{3} \left(f(a) + 4f(\frac{a+b}{2}) + f(b) \right)
\end{equation}
% paragraph Simpson求积公式 (end)
\begin{definition}
  (代数精确度)
\end{definition}
\begin{eg}
  (求求积公式与余项中的待定系数, 并给出代数精度)
  \begin{itemize}
    \item 求积公式中代入$f(x)=1, x, x^2$, 求求积公式中的待定系数
    \item 求积公式中代入$f(x)=x^3$, 观察求积公式两边是否相等
    \item 带余项的求积公式中代入$f(x)=x^3$, 求余项中的待定系数
  \end{itemize}
\end{eg}
\paragraph{插值型求积公式}
\begin{equation}
  I_n(f) = \sum^n_{k=0} A_k f(x_k)
\end{equation}
% paragraph 插值型求积公式 (end)
\begin{theorem}
  (等距节点的插值型求积公式)
\end{theorem}
% subsection 梯形公式和Simpson公式 (end)
\subsection{复化梯形公式和复化Simpson公式}
\paragraph{复化梯形公式}
\begin{equation}
  \begin{aligned}
    T_n(f) &= \frac{h}{2} + \left[ f(a) + 2\sum^{n-1}_{k=1} f(a+kh) + f(b)  \right]\\
    R_{T_n} &= - \frac{(b-a)h^2}{12} f''(\eta), \eta\in(a,b)
  \end{aligned}
\end{equation}
% paragraph 复化梯形公式 (end)
\paragraph{复化Simpson公式}
\begin{equation}
  \begin{aligned}
    S_n(f) &= \frac{h}{3} \left( f(a) + 4\sum^{n-1}_{k=0} f(a+(2k+1)h) + 2\sum^{n-1}_{k=1} f(a+kh) + f(b) \right)\\
    R_{S_n} &= - \frac{(b-a)h^4}{180} f^{(4)}(\eta), \eta\in(a,b)
  \end{aligned}
\end{equation}
% paragraph 复化Simpson公式 (end)
\begin{eg}
  (用复化梯形公式和复化Simpson公式的余项计算需要的节点数)
\end{eg}
\begin{definition}
  ($p$阶收敛)
  \begin{itemize}
    \item 复化梯形公式2阶收敛, 复化Simpson公式4阶收敛.
  \end{itemize}
\end{definition}
\begin{definition}
  (黎曼可积)
\end{definition}
\begin{theorem}
  (复化梯形公式与复化Simpson公式黎曼可积)
\end{theorem}
% subsection 复化梯形公式和复化Simpson公式 (end)
% section 梯形公式和Simpson求积公式 (end)

\section{Gauss求积公式}
\begin{definition}
  (Gauss求积公式)
  $n+1$个互异节点, $n+1$个求积系数, $2n+1$代数精度
\end{definition}
\subsection{Gauss点与正交多项式零点的关系}
\begin{theorem}
  (Gauss点的充要条件)
\end{theorem}
\begin{lemma}
  ($(a,b)$上带权$\rho(x)$的正交多项式$\varphi_{n+1}(x)$的零点就是Gauss点)
\end{lemma}
\begin{eg}
  (求使公式为Gauss求积公式的节点与系数)
  \begin{itemize}
    \item 构造$\{\varphi_j(x)\}$, 用正交性求待定系数 ($\int_0^1 \sqrt{1-x}(x+a)$怎么算?换元)
    \item 用$\{\varphi_j(x)\}$的零点求节点
    \item 用$f(x)=1, x$求Gauss求积公式的待定系数
  \end{itemize}
\end{eg}
% subsection Gauss点与正交多项式零点的关系 (end)
\subsection{常用的Gauss型求积公式}
\subsubsection{Gauss-Legendre公式}
\paragraph{Gauss-Legendre公式}
\begin{equation}
  \int^1_{-1} f(x) \approx \sum^n_{k=0} A_k f(x_k)
\end{equation}
% paragraph Gauss-Legendre公式 (end)
\begin{table}[htbp]
  \centering
  \caption{Gauss-Legendre公式的求积节点和求积系数}
  \begin{tabular}{cccc}
    \toprule
    $n$ & 点数 & $x_k$ & $A_k$ \\
    $1$ & $2$  & $\pm 1/\sqrt{3}$ & $1.0$ \\
    $2$ & $3$  & $\pm 1/\sqrt{3/5},0$ & $5/9, 8/9$ \\
    $3$ & $4$  & $\pm 0.861136, \pm 0.339981$ & $ 0.347855, 0.652145$\\
    $4$ & $5$  & $\pm 0.906180, \pm 0.538469$ & $0.236927, 0.478629, 0.568889$ \\
  \end{tabular}
\end{table}
\begin{eg}
  (构造Gauss-Legendre公式)
  \begin{itemize}
    \item 变量替换$x=(a+b)/2 + (b-a)t/2$
      \begin{equation}
        \begin{aligned}
          \int^b_a f(x)dx &= \frac{b-a}{2} \int^1_{-1} f(\frac{(a+b)}{2} + \frac{b-a}{2}t)dt \\
                          &\approx \frac{b-a}{2} \sum^n_{k=0} A_k f(\frac{(a+b)}{2} + \frac{b-a}{2}t_k)
        \end{aligned}
      \end{equation}
  \end{itemize}
\end{eg}
\paragraph{复化Gauss-Legendre求积公式}
\begin{itemize}
  \item I型
    \begin{equation}
      \int^a_b f(x)dx \approx \frac{h}{2} \sum^{n-1}_{k=0} [f(x_{k+\frac{1}{2}} - \frac{h}{2\sqrt{3}}) + f(x_{k+\frac{1}{2}} + \frac{h}{2\sqrt{3}})]
    \end{equation}
  \item II型
\end{itemize}
% paragraph 复化Gauss-Legendre求积公式 (end)
\begin{eg}
  (用3段4点复化梯形公式, 2段5点复化Simpson公式, 4点Gauss-Legendre公式和2段Gauss-Legendre I型求积公式计算)
\end{eg}
% subsubsection Gauss-Legendre公式 (end)
\subsubsection{Gauss-Chebyshev求积公式}
(求积公式与相应的零点)
% subsubsection Gauss-Chebyshev求积公式 (end)
% subsection 常用的Gauss型求积公式 (end)
\subsection{Gauss公式的余项}
\begin{theorem}
  (Gauss求积公式的余项)
\end{theorem}
% subsection Gauss公式的余项j (end)
\begin{theorem}
  (求积公式数值稳定的定义)
\end{theorem}
\begin{theorem}
  (Gauss求积公式数值稳定)
\end{theorem}
\begin{theorem}
  (Gauss求积公式收敛)
\end{theorem}
\subsection{Gauss求积公式的数值稳定性和收敛性}
% subsection Gauss求积公式的数值稳定性和收敛性 (end)
% section Gauss求积公式 (end)

\section{数值微分}
\begin{itemize}
  \item 对于一阶导数
    \begin{itemize}
      \item 一次差商或线性插值: $h$精度
      \item 中心差商或二次插值: $h^2$精度
      \item 三次样条: $h^3$精度
      \item 隐式差分: $h^4$精度
    \end{itemize}
  \item 对于二阶导数
    \begin{itemize}
      \item 二次中心差商或二次插值: $h^2$精度
      \item 三次样条: $h^2$精度
      \item 隐式差分: $h^4$精度
    \end{itemize}
\end{itemize}
\subsection{Taylor展开法}
\begin{itemize}
  \item 向前差商, 向后差商, 中心差商
  \item 求解各节点上一阶与二阶导数近似值的隐式方法
    \begin{equation}
      \begin{aligned}
        G_1m&=f_1 \\
        G_2M&=f_2
      \end{aligned}
    \end{equation}
\end{itemize}
% subsection Taylor展开法 (end)
\subsection{插值型求导公式}
\begin{itemize}
  \item 插值型求导公式, 余项
  \item 三点公式, 五点公式
\end{itemize}
% subsection 插值型求导公式 (end)
\subsection{三次样条求导}
三次样条插值函数$S(x)$
% subsection 三次样条求导 (end)
% section 数值微分 (end)

\section{外推技巧和自适应技术}
\subsection{外推原理}
\begin{equation}
  F_1(h)=\frac{2F(h/2)-F(h)}{2-1}=F(0)+b_2h^2 + b_3h^3+\cdots
\end{equation}
\paragraph{Richaegson外推法}
\begin{equation}
  \begin{aligned}
    F(h)-F(0)&=a_p h^p + O(h^s), s>p \\
    F_1(h) &= \frac{q^pF(h/q)-F(h)}{q^p-1}
  \end{aligned}
\end{equation}
\begin{theorem}
\end{theorem}
% paragraph Richaegson外推法 (end)
% subsection 外推原理 (end)
\subsection{数值微分的外推算法}
\begin{eg}
  (中心差商的外推)
  \begin{equation}
    \begin{aligned}
      G(x,h) &= \frac{f(x+h)-f(x-h)}{2h} \\
      G_1(h)&=G(x_n,h) \\
      G_{k+1}(h)&=\frac{4^k G_k(h/2)-G_k(h)}{4^k -1}
    \end{aligned}
  \end{equation}
\end{eg}
% subsection 数值微分的外推算法 (end)
\subsection{数值积分的Romberg算法}
\begin{eg}
  (Romberg算法)
  \begin{equation}
    \begin{cases}{}
      T^{(0)}_1 &= \frac{b-a}{2}[f(a)+f(b)] \\
      T^{(l)}_1 &= \frac{1}{2} [T^{(l-1)}_1 + \frac{b-a}{2^{l-1}} \sum^{2^{l-1}}_{i=1} f(a+(2i-1)\frac{b-a}{2^l})] \\
      T^{(k-1)}_{m+1} &= \frac{4^m T^{(k)}_m - T^{(k-1)}_m}{4^m-1}
    \end{cases}
  \end{equation}

\end{eg}
% subsection 数值积分的Romberg算法 (end)
\subsection{自动变步长Simpson方法和自适应Simpson方法}
\begin{itemize}
  \item 自动变步长Simpson方法
  \item 自适应Simpson方法
\end{itemize}

% subsection 自动变步长Simpson方法和自适应Simpson方法 (end)
% section 外推技巧和自适应技术 (end)

\section{应用例题}
\begin{eg}
\end{eg}
% section 应用例题 (end)
\end{document}
