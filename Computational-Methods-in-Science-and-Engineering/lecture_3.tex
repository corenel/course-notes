% !TEX program = xelatex
% !TEX encoding = UTF-8 Unicode

\documentclass[twoside]{article}
\setlength{\oddsidemargin}{0.25 in}
\setlength{\evensidemargin}{-0.25 in}
\setlength{\topmargin}{-0.6 in}
\setlength{\textwidth}{6.5 in}
\setlength{\textheight}{8.5 in}
\setlength{\headsep}{0.75 in}
\setlength{\parindent}{0 in}
\setlength{\parskip}{0.1 in}

%
% ADD PACKAGES here:
%

\usepackage{xeCJK}
\usepackage{amsmath,amsfonts,amsthm,graphicx}
\usepackage{mathtools}
\usepackage{color}   %May be necessary if you want to color links
\usepackage{hyperref}
\usepackage{enumitem}
\usepackage{mleftright}
\hypersetup{
    colorlinks=true, %set true if you want colored links
    linktoc=all,     %set to all if you want both sections and subsections linked
    linkcolor=red,  %choose some color if you want links to stand out
}

%
% The following commands set up the lecnum (lecture number)
% counter and make various numbering schemes work relative
% to the lecture number.
%
\newcounter{lecnum}
\renewcommand{\thepage}{\thelecnum-\arabic{page}}
\renewcommand{\thesection}{\thelecnum.\arabic{section}}
\renewcommand{\theequation}{\thelecnum.\arabic{equation}}
\renewcommand{\thefigure}{\thelecnum.\arabic{figure}}
\renewcommand{\thetable}{\thelecnum.\arabic{table}}

%
% The following macro is used to generate the header.
%
\newcommand{\lecture}[4]{
   \pagestyle{myheadings}
   \thispagestyle{plain}
   \newpage
   \setcounter{lecnum}{#1}
   \setcounter{page}{1}
   \noindent
   \begin{center}
   \framebox{
      \vbox{\vspace{2mm}
    \hbox to 6.28in { {\bf 科学与工程计算方法
        \hfill Winter 2017} }
       \vspace{4mm}
       \hbox to 6.28in { {\Large \hfill Lecture #1: #2  \hfill} }
       \vspace{2mm}
       \hbox to 6.28in { {\it Lecturer: #3 \hfill Scribes: #4} }
      \vspace{2mm}}
   }
   \end{center}
   \markboth{Lecture #1: #2}{Lecture #1: #2}
}
%
% Convention for citations is authors' initials followed by the year.
% For example, to cite a paper by Leighton and Maggs you would type
% \cite{LM89}, and to cite a paper by Strassen you would type \cite{S69}.
% (To avoid bibliography problems, for now we redefine the \cite command.)
% Also commands that create a suitable format for the reference list.
\renewcommand{\cite}[1]{[#1]}
\def\beginrefs{\begin{list}%
        {[\arabic{equation}]}{\usecounter{equation}
         \setlength{\leftmargin}{2.0truecm}\setlength{\labelsep}{0.4truecm}%
         \setlength{\labelwidth}{1.6truecm}}}
\def\endrefs{\end{list}}
\def\bibentry#1{\item[\hbox{[#1]}]}

%Use this command for a figure; it puts a figure in wherever you want it.
%usage: \fig{NUMBER}{SPACE-IN-INCHES}{CAPTION}
\newcommand{\fig}[3]{
    \vspace{#2}
    \begin{center}
        Figure \thelecnum.#1:~#3
    \end{center}
}
% Use these for theorems, lemmas, proofs, etc.
\newtheorem{theorem}{定理}[section]
\newtheorem{algo}{算法}[section]
\newtheorem{lemma}{引理}[section]
\newtheorem{proposition}{命题}[section]
\newtheorem{claim}{Claim}[section]
\newtheorem{corollary}{推论}[section]
\newtheorem{definition}{定义}[section]
\newtheorem{eg}{例}[section]
% \newenvironment{proof}{{\bf Proof:}}{\hfill\rule{2mm}{2mm}}

% **** IF YOU WANT TO DEFINE ADDITIONAL MACROS FOR YOURSELF, PUT THEM HERE:
\newcommand{\norm}[1]{\left\lVert#1\right\rVert}

\begin{document}
%FILL IN THE RIGHT INFO.
%\lecture{**LECTURE-NUMBER**}{**DATE**}{**LECTURER**}{**SCRIBE**}
\lecture{3}{线性代数方程组的数值解法}{Zhitao Liu}{Yusu Pan}
%\footnotetext{These notes are partially based on those of Nigel Mansell.}

% **** YOUR NOTES GO HERE:

% Some general latex examples and examples making use of the
% macros follow.
%**** IN GENERAL, BE BRIEF. LONG SCRIBE NOTES, NO MATTER HOW WELL WRITTEN,
%**** ARE NEVER READ BY ANYBODY.

\section{引言}
\begin{itemize}
  \item \textit{大型线性代数方程组} (P61)
    \begin{equation}
      Ax=b
    \end{equation}
  \item 系数矩阵$A$, 右端向量$b$和解向量$x$ (P61)
  \item Gram法则: 需要$n!\times(n-1)$次乘法
  \item \textit{直接法}: 将原式化为上三角方程组
    \begin{equation}
      Ux=L^{-1}b
    \end{equation}
    \begin{itemize}
      \item 高斯消去法和它的变种, 直接三角分解法
      \item 对于中小型方程组, 常选用直接法, 因为直接法能在预定的计算步骤内求得精确解 (不计舍入误差), 可靠且效率高.
      \item 对于高阶的, 特别是大型稀疏方程组, 由于直接法存储量和计算量太大, 通常迭代法更具竞争力.
    \end{itemize}
  \item \textit{基本迭代法}, 如Jacobi迭代法和Gauss-Seidel迭代法
\end{itemize}
% section 引言 (end)

\section{高斯消去法}
\subsection{顺序消去过程和矩阵的LU分解}
\begin{itemize}
  \item \textit{高斯消去法的消元过程}: P62-64
    \begin{itemize}
      \item \textit{消元因子}: $l_{ik}$
      \item \textit{主元素}: $a^{(k)}_{kk}$
      \item 高斯消元回代过程均要求主元素$a^{(k)}_{kk}\ne 0$
    \end{itemize}
  \item \textit{高斯消去法的回代过程}: P64
  \item \textit{矩阵的LU分解}: P64-65
    \begin{equation}
      A=LU
    \end{equation}
    \begin{equation}
      \det{A} = \prod^n_{i=1} a^{(i)}_{ii}
    \end{equation}
\end{itemize}
\begin{eg}
  (求系数矩阵$A$的三角分解并计算$\det{A}$)
  P66
\end{eg}
% subsection 顺序消去过程和矩阵的LU分解 (end)
\subsection{可行性和计算量}
\begin{theorem}
  如果矩阵$A$的各阶顺序主子是式均不为零, 即$D_k=\det{A}\ne0$, 则$a^{(k)}_{kk}\ne 0$
\end{theorem}
\begin{itemize}[leftmargin=2em]
  \item \textit{$A$对称正定} $\Rightarrow$ $a^{(k)}_{kk}> 0$
  \item \textit{$A$严格对焦占优} $\Rightarrow$  $a^{(k\ne}_{kk}> 0$
  \item \textit{计算量}: 当$n$很大时, 略去$n$的低次项, 则加减乘除运算总次数以$\fra{2n^3}{3}$的渐近速度增长
\end{itemize}
% subsection 可行性和计算量 (end)
\subsection{数值的稳定性: 选主元}
\begin{itemize}
  \item \textit{零主元}或\textit{小主元}会导致消元不可行或数值不稳定
  \item \textit{选主元思想}: 在第$k$列元素$a^{(k)}_{kk}$之下的所有元素中选一个绝对值最大的元素作为主元素, 使消元因子$|l_{ik}|<1$, 达到抑制舍入误差的作用, 使算法基本稳定. (P69)
  \item \textit{列主元消去法}: 通过换行来选主元. 一个矩阵的两行交换, 其行列式反号.
  \item \textit{完全选主元法}: 在$A^{(k)}$中右下角的子矩阵中选取绝对值最大的元素作为主元素.
  \item 高斯选主元法实现矩阵的三角分解: 排列阵$p$ (P72)
    \begin{equation}
      A=pLU
    \end{equation}
\end{itemize}
\begin{eg}
  (使用高斯列主元消去法求解线性方程组)
\end{eg}
\begin{eg}
  (使用高斯列主元消去法求矩阵的三角分解)
\end{eg}
% subsection 数值的稳定性: 选主元 (end)
% section 高斯消去法 (end)

\section{矩阵的直接三角分解法}
\begin{itemize}
  \item \textit{三对角形方程组}
    \begin{equation}
      Ax=f
    \end{equation}
    \begin{equation}
      A =
      \begin{bmatrix}
        b_1 & c_1    &        &        & \\
        a_2 & b_2    & c_2    &        & \\
            & \ddots & \ddots & \ddots & \\
            &        & a_{n-1}& b_{n-1}& c_{n-1} \\
            &        &        & a_n    & b_n \\
      \end{bmatrix}
    \end{equation}
    \begin{equation}
      L =
      \begin{bmatrix}
        1   &        &        &        & \\
        l_2 & 1      &        &        & \\
            & l_3    & \ddots &        & \\
            &        & \ddots & 1      & \\
            &        &        & l_n    & 1 \\
      \end{bmatrix}
      , U =
      \begin{bmatrix}
        u_1 & d_1    &        &        & \\
            & u_2    & d_2    &        & \\
            &        & \ddots & \ddots & \\
            &        &        & u_{n-1}&d_{n-1} \\
            &        &        &        & u_n \\
      \end{bmatrix}
    \end{equation}
\end{itemize}
\subsection{三对角形方程组的追赶法}
\begin{equation}
  \begin{cases}
    d_i=c_i \\
    u_1=b_1 \\
    l_i=\frac{a_i}{u_{i-1}} \\
    u_i=b_i-l_i c_{i-1}
  \end{cases}
\end{equation}
\begin{equation}
  \begin{cases}{}
    y_1=f_1\\
    y_i=f_i-l_i y_{i-1}\\
  \end{cases}
\end{equation}
\begin{equation}
  \begin{cases}{}
    x_n = \frac{y_n}{u_n} \\
    x_i = \frac{y_i-c_ix_{i+1}}{u_i}\\
  \end{cases}
\end{equation}
\begin{itemize}
  \item 整个求解过程仅需$5n-4$次乘除和$3(n-1)$次加减运算,共$8n-7$次运算
  \item 仅需$4$个一维数组
\end{itemize}
\begin{theorem}
  (追赶法可行的充要条件)
\end{theorem}
\begin{eg}
  (用追赶法求解三对角方程组$Ax=f$)
\end{eg}
% subsection 三对角形方程组的追赶法 (end)
\subsection{对称正定的Cholesky分解法}
\begin{theorem}
  \begin{equation}
    \begin{aligned}
      A&=LL^T \\
      l_{jj}&=\sqrt{a_{jj}-\sum^{j-1}_{k=1} l^2_{jk}}\\
      l_{ij}&=\frac{a_{ij}-\sum^{j-1}_{k=1} l_{ik} l_{jk}}{l_{jj}}
    \end{aligned}
  \end{equation}
\end{theorem}
\paragraph{Cholesky法 (或平方根法)}
\begin{equation}
  \begin{cases}
    Ly=b \\
    L^Tx=y
  \end{cases}
\end{equation}
计算公式为:
\begin{equation}
  \begin{aligned}
    y_1&=\frac{b_1}{L_{11}} \\
    y_i&=\frac{(b_i-\sum^{i-1}_{k=1} l_{ik} y_k)}{l_{ii}} \\
  \end{aligned}
\end{equation}
\begin{equation}
  \begin{aligned}
    x_n&=\frac{y_n}{l_{nn}} \\
    x_i&=\frac{(y_i-\sum^n_{k=i+1}l_{ki}x_k)}{l_{ii}}
  \end{aligned}
\end{equation}
\begin{itemize}
  \item 计算量比高斯消去法减少了一半
  \item 通常是数值稳定的, 不必选主元
\end{itemize}
\begin{eg}
  (用Cholesky方法解方程组$Ax=b$)
  \begin{itemize}
    \item 对称正定: $A^T=A$, $D_k(A)>0$
    \item $A=LL^T$
    \item $Ly=b$
    \item $L^Tx=y$
  \end{itemize}
\end{eg}
% paragraph Cholesky法 (end)
% subsection 对称正定的Cholesky分解法 (end)
% section 矩阵的直接三角分解法 (end)

\section{方程组的性态, 条件数}
\begin{definition}
  (病态矩阵, 病态方程组与良态方程组)
\end{definition}
\begin{definition}
  (条件数)
  \begin{equation}
    cond(A)=\|A^{-1}\| \|A\|
  \end{equation}
\end{definition}
\paragraph{常用的条件数}
\begin{equation}
  \begin{aligned}
    cond_\infty(A)&=\|A^{-1}\|_\infty \|A\|_\infty \\
    cond_1(A)&=\|A^{-1}\|_1 \|A\|_1 \\
    cond_2(A)&=\|A^{-1}\|_2 \|A\|_2 = \sqrt{\frac{\lambda_{\max}(A^TA)}{\lambda_{\min}(A^TA)}}
  \end{aligned}
\end{equation}
\begin{itemize}[leftmargin=2em]
  \item \textit{1范数}: 列和范数
  \item \textit{$\infty$范数}: 行和范数
\end{itemize}
% paragraph 常用的条件数 (end)
\begin{eg}
  (计算$\infty$-条件数, 并说明$\delta b$对解向量的影响)
\end{eg}
\begin{theorem}
  (事前误差估计式)
  \begin{equation}
    \frac{\|\delta x\|}{\|x\|} \le \cdots
  \end{equation}
\end{theorem}
\begin{corollary}
\end{corollary}
\begin{corollary}
\end{corollary}
\begin{theorem}
  (事后误差估计式)
  \begin{equation}
    \cdots \le \frac{\|\tilde{x} - x\|}{\|x\|} \le \cdots
  \end{equation}
\end{theorem}
\begin{itemize}
  \item 说明只有当$cond(A)\approx 1$时, 方程组余量的相对误差是$\delta x$相对误差的很好度量.
\end{itemize}
% section 方程组的性态, 条件数 (end)

\section{大型方程组的迭代方法}
(不考)
\subsection{Jacobi迭代和Gauss-Seidel迭代法}
% subsection Jacobi迭代和Gauss-Seidel迭代法 (end)
\subsection{迭代法的收敛性和收敛速度}
% subsection 迭代法的收敛性和收敛速度 (end)
\subsection{Jacobi迭代和Gauss-Seidel迭代法的收敛性判定}
% subsection Jacobi迭代和Gauss-Seidel迭代法的收敛性判定 (end)
\subsection{分块迭代法}
% subsection 分块迭代法 (end)
% section 大型方程组的迭代方法 (end)
\end{document}

